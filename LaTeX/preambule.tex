\documentclass[10pt,leqno]{scrartcl}

% \usepackage[francais]{babel} %% what is the difference with french, or
                 %% frenchb?

\usepackage{a4wide} %% maybe not so good for the
            %% non-europeans. Should we use the geometry package?

%% in order to use accented characters.

%\usepackage{isolatin1}

\usepackage{theorem}

\usepackage{amsmath}

%% for other fonts like \mathbb and \mathfrak; amssymb contains
%% amsfonts.... So that gives more choice for the typists.

\usepackage{amssymb}

%% for simple rectangular diagrams

\usepackage{amscd}

%% for the more complicated diagrams one can use the XY-pic package.

\usepackage[all]{xy}

%% Why not use the times fonts

\usepackage{times}

%% for a maybe nicer script font

\usepackage{mathrsfs}

%% We want fancy headings

\usepackage{fancyheadings}
\usepackage{hyperref}

%% This comes from the LaTeX companion, pages 92-93.

\newcommand{\clearemptydoublepage}
{\newpage{\pagestyle{empty}\cleardoublepage}}

%% We want the chapter numbers in large Roman numbers.

% \renewcommand{\thechapter}{\Roman{chapter}}

%% we want the numbering of the sections and subsections to be as in
%% the book, i.e., without the chapter number in front of them

\renewcommand{\thesection}{\arabic{section}}
\renewcommand{\thesubsection}{\thesection.\arabic{subsection}}
\renewcommand{\thesubsubsection}{\thesubsection.\arabic{subsubsection}}
\renewcommand{\theparagraph}{\thesubsubsection.\arabic{paragraph}}
\renewcommand{\theequation}{\arabic{equation}}

%% We define the theorem environments

%% First those that have their text underlined (becomes slanted).

%%\theorembodyfont{\sl}
\newtheorem{prop}[subsection]{Proposition}
\newtheorem{lemma}[subsection]{Lemma}
\newtheorem{thm}[subsection]{Theorem}
\newtheorem{cor}[subsection]{Corollary}

\newtheorem{sublemma}[subsubsection]{Lemma}
\newtheorem{subthm}[subsubsection]{Theorem}
\newtheorem{subprop}[subsubsection]{Proposition}
\newtheorem{subcor}[subsubsection]{Corollary}


%% And those that do not have their text underlined.

\theorembodyfont{\rmfamily}

\newtheorem{defi}[subsection]{Definition}
\newtheorem{rmk}[subsection]{Remark}
\newtheorem{exemple}[subsection]{Exemple}
\newtheorem{remarques}[subsection]{Remarques}
\newtheorem{fait}[subsection]{Fait}

\newtheorem{subdefi}[subsubsection]{Definition}
\newtheorem{subrmk}[subsubsection]{Remark}
\newtheorem{subex}[subsubsection]{Example}
\newtheorem{subremarques}[subsubsection]{Remarques}
\newtheorem{subnotation}[subsubsection]{Notation}
\newtheorem{subfact}[subsubsection]{Fact}

%% And a *-form:

\newtheorem{remarquesstar}{Remarques}
\renewcommand{\theremarquesstar}{}
\newtheorem{exemplesstar}{Exemples}
\renewcommand{\theexemplesstar}{}
\newtheorem{remarquestar}{Remarque}
\renewcommand{\theremarquestar}{}
\newtheorem{anything}{}


%% Preuves

\newenvironment{proof}{\vspace{0.3cm}\noindent
\emph{Proof.}}{\hfill $\square$ \vspace{0.3cm}}
\newenvironment{proofarg}[1]{\vspace{0.3cm}\noindent
\emph{Proof of \nobreakspace#1.}}{\hfill
$\square$ \vspace{0.3cm}}

%% for the equations: this will not work for every chapter since the
%% numbering depends heavily on the chapter

\numberwithin{equation}{subsubsection}

%% redefine \subsection so that the text following it starts on the
%% same line.

% \makeatletter
% \renewcommand{\subsection}
% {\@startsection {subsection}{2}{\z@ }{-3.25ex\@plus -1ex \@minus
% -.2ex} {-1.5ex}{\normalfont \normalsize \bfseries }} \makeatother


%% redefine \subsubsection so that the text following it starts on the
%% same line.

% \makeatletter
% \renewcommand{\subsubsection}
% {\@startsection {subsubsection}{3}{\z@ }{-3.25ex\@plus -1ex
% \@minus -.2ex} {-1.5ex}{\normalfont \normalsize \bfseries }}
% \makeatother

%% redefine \paragraphsection so that the text following it starts on the
%% same line.

% \makeatletter
% \renewcommand{\paragraph}
% {\@startsection {paragraph}{4}{\z@ }{-3.25ex\@plus -1ex \@minus
% -.2ex} {-1.5ex}{\normalfont \normalsize \bfseries }} \makeatother

%% Some commands for the typesetting choices....

%% The choice for the script font. Some underlined math symbols become
%% script. For example: \mathcal{O} for structure sheaves.

\renewcommand{\mathcal}{\mathscr}

%% The choice for the black board bold font.

\renewcommand{\Bbb}{\mathbb}

%% The choice for the gothic font. These are handwritten in the typed
%% text.

\newcommand{\goth}{\mathfrak}

%% make the old font selection commands from LaTeX 2.09 work as the
%% ones we want, but only in math mode.

\renewcommand{\rm}{\mathrm}
\renewcommand{\it}{\mathit}
\renewcommand{\bf}{\mathbf}
\renewcommand{\cal}{\mathcal}

%% Macros. See AMSLaTex user's guide for explanations.

%% Ensembles

\newcommand{\C}{\mathbb{C}}
\newcommand{\Nat}{\mathbb{N}}
\newcommand{\R}{\mathbb{R}}
\newcommand{\Q}{\mathbb{Q}}
\newcommand{\Z}{\mathbb{Z}}
\newcommand{\Fi}{\mathbb{F}}
\newcommand{\Af}{\mathbb{A}_f}
\newcommand{\Aff}{\mathbb{A}}
\newcommand{\Proj}{\mathbb{P}}

%% Cohomologie

\newcommand{\Hp}{{}^p\mathrm{H}}
\renewcommand{\H}{\mathrm{H}}
\newcommand{\HH}{\mathcal{H}}
\newcommand{\HyH}{\mathbb{H}}
\newcommand{\Hc}{\check{\mathrm{H}}}

%% Hom

\DeclareMathOperator{\Hom}{Hom}
\DeclareMathOperator{\Ext}{Ext}
\DeclareMathOperator{\Homf}{\underline{Hom}}
\DeclareMathOperator{\Extf}{\underline{Ext}}

%% Groupes

\newcommand{\SD}{\mathbb{S}}
\newcommand{\GL}{\bf {GL}}
%%\newcommand{\SL}{\bf SL}
\newcommand{\GSp}{\bf GSp}
\newcommand{\Sp}{\bf Sp}
\newcommand{\GU}{\bf GU}
\newcommand{\SU}{\bf SU}
\newcommand{\Gr}{\mathbb{G}}
\newcommand{\Pa}{\bf P}
\newcommand{\QP}{\bf Q}
\newcommand{\RP}{\bf R}
\newcommand{\B}{\bf B}
\newcommand{\N}{\bf N}
\newcommand{\Le}{\bf L}
\newcommand{\Se}{\bf S}
\newcommand{\Ar}{\mathrm{A}}
\newcommand{\Hr}{\mathrm{H}}

%% Faisceaux

\newcommand{\F}{\mathcal{F}}
\newcommand{\Gf}{\mathcal{G}}
\newcommand{\Of}{\mathcal{O}}
\newcommand{\Hf}{\mathcal{H}}
\newcommand{\Kf}{\mathcal{K}}
\newcommand{\Mf}{\mathcal{M}}
\newcommand{\Bf}{\mathcal{B}}
\newcommand{\Uf}{\mathcal{U}}
\newcommand{\Cf}{\mathcal{C}}
\newcommand{\Pf}{\mathcal{P}}

%% Divers

\newcommand{\Ab}{\mathbf{Ab}}  %% category of abelian groups
\newcommand{\adj}{\mathrm{adj}} 
\newcommand{\aff}{\mathrm{aff}} 
\newcommand{\as}{\underline{a}}
\DeclareMathOperator{\can}{can}
\DeclareMathOperator{\card}{card} 
\DeclareMathOperator{\Chow}{Chow}
\DeclareMathOperator{\codim}{codim}
\DeclareMathOperator{\Coker}{Coker}
\DeclareMathOperator{\D}{D} %% cat�gorie d�riv�e
\DeclareMathOperator{\DF}{DF} %% cat�gorie d�riv�e filtr�e
\DeclareMathOperator{\DM}{DM} %% motifs mixtes
\newcommand{\DP}{{}^{w}\D}
\newcommand{\Dp}{{}^{p}\D}
\DeclareMathOperator{\diag}{diag}
\DeclareMathOperator{\E}{E} %% spectral sequence (I.2.6)
\newcommand\equv{\Longleftrightarrow}
\newcommand{\et}{\mathrm{\acute{e}t}} %% plus accent
\newcommand{\fl}{\longrightarrow}
\newcommand{\ra}{\longrightarrow}
\newcommand{\Ra}{\Longrightarrow}
\def\flnom#1{\stackrel{#1}{\fl}}
\newcommand{\fle}{\longmapsto}
\DeclareMathOperator{\Frac}{Frac}
\DeclareMathOperator{\Gra}{Gr} %% graded
\DeclareMathOperator{\Gal}{Gal} %% for Galois group
\newcommand{\id}{\mathrm{id}} %% morphisme identit� 
\renewcommand{\Im}{\mathrm{Im}} %% replaces old \Im (Imaginary part
                 %% of complex number)
\DeclareMathOperator{\Ind}{Ind}
\newcommand{\iso}{\stackrel{\sim}{\fl}}
\DeclareMathOperator{\Ker}{Ker}
\newcommand{\lin}{\ell} 
\newcommand{\Lf}{\mathcal{L}} %% syst�me local
\newcommand{\Ltimes}{\otimes^L}
\newcommand{\LS}{\mathrm{LS}}
\newcommand{\PLS}{\mathrm{PLS}}
\newcommand{\SL}{\mathrm{SL}}
\newcommand{\M}{\mathcal{M}} %% cat�gorie de faisceaux pervers
\DeclareMathOperator{\Ob}{\mathrm{Ob}}
\newcommand{\op}{\mathrm{op}} 
\newcommand\oQ{\overline{\Q}}
\DeclareMathOperator{\Perv}{Perv}
\DeclareMathOperator{\Pro}{Pro}
\DeclareMathOperator{\PSh}{PSh}
\DeclareMathOperator{\DR}{R}  %% for right-derivation of functors
\DeclareMathOperator{\DL}{L} %% for left-derivation of functors
\DeclareMathOperator{\Pre}{Pre}
\newcommand\quash[1]{}
\newcommand{\real}{\mathrm{real}} %% foncteur r�alisation
\newcommand{\red}{\mathrm{r\acute{e}d}} %% for reduced closed subscheme
\newcommand{\Sf}{\mathcal{S}}
\newcommand{\Sch}{\mathbf{Sch}}
\newcommand{\Sets}{\mathrm{Sets}} 
\DeclareMathOperator{\Sh}{Sh}
\newcommand{\ssi}{si et seulement si}
\newcommand{\sous}{\setminus}
\DeclareMathOperator{\Spec}{Spec}
\DeclareMathOperator{\supp}{supp} %% for the support of a sheaf
\DeclareMathOperator{\Sym}{Sym} %% puissances sym�triques
\newcommand{\tame}{\mathrm{t}}
\newcommand{\TR}{\mathfrak{TR}}
\newcommand{\til}{\widetilde}
\newcommand{\U}{\mathcal{U}}
\newcommand{\ungras}{1\!\!\mkern -1mu1}
\newcommand{\var}{\mathrm{var}}
\newcommand{\X}{\mathcal{X}}
\newcommand{\Y}{\mathcal{Y}}
\newcommand{\Zar}{\mathrm{Zar}} 

%% For comments to myself.

% \newcommand\footnotevar[1]{\footnote{#1}}
\newcommand\footnotevar[1]{}

%% In order to prevent line breaks in formulas.

\relpenalty=10000 \binoppenalty=10000


%% In order to avoid too many over hboxes and so on for the moment.

\sloppy


%% The following is an attempt to make the table of contents better,
%% as the Roman numbers were too wide.

% \makeatletter

% \renewcommand*\l@chapter[2]{%
%   \ifnum \c@tocdepth >\m@ne
%     \addpenalty{-\@highpenalty}%
%     \vskip 1.0em \@plus\p@
%    \setlength\@tempdima{2.5em}%
%    \begingroup
%      \parindent \z@ \rightskip \@pnumwidth
%      \parfillskip -\@pnumwidth
%      \leavevmode \bfseries
%      \advance\leftskip\@tempdima
%      \hskip -\leftskip
%      #1\nobreak\hfil \nobreak\hb@xt@\@pnumwidth{\hss #2}\par
%      \penalty\@highpenalty
%    \endgroup
%  \fi}
%
% \renewcommand{\l@section}
% {\@dottedtocline {1}{2.5em}{2.3em}}

% \makeatother


%% We make the space between paragraphs bigger (note: it is
%% temporarily changed back in the table of contents)

\setlength{\parskip}{0.5\baselineskip}
