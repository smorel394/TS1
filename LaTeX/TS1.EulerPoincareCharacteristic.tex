\section{Euler-Poincaré characteristic of a finite simplicial complex}

We fix a set $\alpha$ and an abstract simplicial complex $K$ on $\alpha$.

\subsection{Definition}

\begin{subdefi}[FacesFinset]
If $K$ is a finite complex, we define the finite set of faces of $K$.

\end{subdefi}

\begin{subdefi}[FacetsFinset]
If $K$ is a finite complex, we define the finite set of facets of $K$.

\end{subdefi}

\begin{subdefi}[EulerPoincareCharacteristic]
If $K$ is a finite complex, its Euler-Poincaré characteristic is the sum over all faces $s$ of $K$ of
$(-1)^{card(s)-1}$.

\end{subdefi}

\begin{sublemma}[EulerPoincareCharacteristic\_ext]
If $K$ and $L$ are finite abstract simplicial complexes with equal sets of faces, then their Euler-Poincaré
characteristic are equal.

\end{sublemma}

\subsection{The case of decomposable complexes}

\begin{subdefi}[$\pi_0$Facets]
Let $(R,DF)$ be a decomposition of $K$. We define the set of $\pi_0$ facets of $K$, as a set of finite subsets of $\alpha$.

\end{subdefi}

\begin{subdefi}[HomologyFacets]
Let $(R,DF)$ be a decomposition of $K$. We define the set of homology facets of $K$, as a set of finite subsets of $\alpha$.

\end{subdefi}

\begin{sublemma}[$\pi_0$Facets\_finite]
Let $(R,DF)$ be a decomposition of $K$, and suppose that $K$ is finite. Then the set of $\pi_0$ facets of $K$ is finite.

\end{sublemma}

\begin{sublemma}[HomologyFacets\_finite]
Let $(R,DF)$ be a decomposition of $K$, and suppose that $K$ is finite. Then the set of homology facets of $K$ is finite.

\end{sublemma}

We now introduce some auxiliary definitions that we will need in the calculation.

\begin{subdefi}[DFe]
If $DF$ is a map from the set of faces of $K$ to the set of facets of $K$, we extend $DF$ to a map $DFe$ from the set of finite
subsets of $\alpha$ to itself, by sending a finite subset $s$ to $DF(s)$ if $s$ is a face of $K$, and to $\varnothing$ otherwise.

\end{subdefi}

\begin{subdefi}[Quotient\_DFe\_to\_finset]
Let $DF$ be a map from the set of faces of $K$ to the set of facets of $K$. We define a map from the quotient of $\alpha$ by
the equivalence relation $\Ker(DFe)$ to the set of finite subsets of $\alpha$ by sending the equivalence class of a finite
set $s$ to $DFe(s)$.

\end{subdefi}

\begin{sublemma}[Quotient\_DFe\_to\_finset\_is\_facet\_aux \& Quotient\_DFe\_to\_finset\_is\_facet]
Let $DF$ be a map from the set of faces of $K$ to the set of facets of $K$. Then the map from $\alpha/\Ker(DFe)$ to the set of finite
subsets of $\alpha$ sends the class of a face of $K$ to a facet of $K$.

\end{sublemma}

\begin{subdefi}[DecompositionInterval']
If $R$ is a map from the set of facets of $K$ to the set of finite subsets of $\alpha$ and $s$ a facet of $K$, we define the
interval $[R(s),s]$ as a finite set of nonempty finite subsets of $\alpha$.

\end{subdefi}

\begin{sublemma}[ComparisonIntervals]
Let $(R,DF)$ be a decomposition of $K$, $s$ be a facet of $K$ and $t$ be a finite subset of $\alpha$.
Then $t$ is in the decomposition interval of the previous definition if and only if it is a face of $K$ and a member of the 
decomposition interval defined by $s$.

\end{sublemma}

\begin{subdefi}[Sum\_on\_DecompositionInterval]
Let $R$ be a map from the set of facets of $K$ to the set of finite subsets of $\alpha$ and $s$ be a finite subset of $\alpha$.
We define a "sum on the decomposition interval corresponding to $s$" in the following way: if $s$ is a facet of $K$, then it is the
sum on the elements $t$ of the decomposition interval of $(-1)^{card(t)-1}$, otherwise it is $0$.

\end{subdefi}

\begin{sublemma}[ComparisonFunctionsonQuotient]
Ler $(R,DF)$ be a decomposition of $K$ and $x$ be an element of $\alpha/\Ker(DFe)$. Suppose that $K$ is a finite complex and
that $x$ is the class of a face of $K$. Then the sum over $t$ in the class $x$ of $(-1)^{card(t)-1}$ is equal to the
"sum on the decomposition interval" function applied to the image of $x$ by the quotient of the map $DFe$.

\end{sublemma}

\begin{sublemma}[Quotient\_DFe\_to\_finset\_inj]
Let $DF$ be a map from the set of faces of $K$ to the set of facets of $K$, and suppose that $K$ is a finite complex.
Then the map from $\alpha/\Ker(DFe)$ to the set of finite subsets of $\alpha$ defined by $DFe$ is injective.
\footnote{Why do we need $K$ to be finite ? Surely this is a very general fact. Same remark about the next lemma.
(I know why the condition is there in the Lean file, it's because I use the "finset" versions of the sets of faces and facets, but this
is not a good reason.)}
\end{sublemma}

\begin{sublemma}[Quotient\_DFe\_to\_finset\_surj]
Let $(R,DF)$ be a decomposition of $K$, and suppose that $K$ is a finite complex.
Then every facet of $K$ is in the image of the map from $\alpha/\Ker(DFe)$ to the set of finite subsets of $\alpha$ defined by $DFe$.
\end{sublemma}

\begin{subdefi}[BoringFacets]
Let $(R,DF)$ be a decomposition of $K$. The set of boring facets is the set of finite subsets of $\alpha$ that are facets of $K$ but
are neither $\pi_0$ facets nor homology facets.

\end{subdefi}

\begin{sublemma}[BoringFacets\_finite]
Let $(R,DF)$ be a decomposition of $K$. If $K$ is finite, then the set of boring facets of $K$ is finite.

\end{sublemma}

\begin{sublemma}[every\_facet\_is\_boring\_or\_interesting \& boring\_is\_not\_interesting]
Let $(R,DF)$ be a decomposition of $K$, and suppose that $K$ is finite. Then the set of facets of $K$, seen as a finite set of
finite subsets of $K$, is the disjoint union of the set of boring facets and the set of facets that are $\pi_0$ or homology facets.

\end{sublemma}

\begin{sublemma}[pi0\_and\_homology\_are\_disjoint]
Let $(R,DF)$ be a decomposition of $K$, and suppose that $K$ is finite. Then the sets of $\pi_0$ and homology facets of $K$, seen
as finite sets of finite subsets of $\alpha$, are disjoint.

\end{sublemma}


\begin{sublemma}[AlternatingSumPowerset]
Let $s$ be a nonempty finite subset of $\alpha$. Then 
the sum of the function $t\mapsto (-1)^{card(t)}$ on the powerset of $s$ is equal to $0$.

\end{sublemma}

\begin{sublemma}[Sum\_on\_FinsetInterval1]
Let $s,t$ be finite subsets of $\alpha$ such that $s\subsetneq t$. Then 
the sum of the function $x\mapsto (-1)^{card(x)}$ on the interval $[s,t]$ is equal to $0$.

\end{sublemma}

\begin{sublemma}[Sum\_on\_FinsetInterval1]
Let $s$ be a nonempty finite subset of $\alpha$. Then 
the sum of the function $x\mapsto (-1)^{card(x)-1}$ on the interval $]\varnothing,s]$ is equal to $1$.

\end{sublemma}

\begin{sublemma}[BoringFacet\_image\_by\_R]
Let $(R,DF)$ be a decomposition of $K$ and $s$ be a boring facet of $K$. Then $R(s)$ is not empty and not equal to $s$.

\end{sublemma}

\begin{sublemma}[Sum\_on\_DecompositionInterval\_BoringFacet]
Let $(R,DF)$ be a decomposition of $K$ and $s$ be a boring facet of $K$. Then the image of $s$ by the sum of the decomposition interval function 
is equal to $0$.

\end{sublemma}

\begin{sublemma}[$\pi_0$Facet\_interval]
Let $(R,DF)$ be a decomposition of $K$ and $s$ be a $\pi_0$ facet of $K$. Then the decomposition interval defined by $s$ is equal to
$]\varnothing,s]$.

\end{sublemma}

\begin{sublemma}[Sum\_on\_DecompositionInterval\_$\pi_0$Facet]
Let $(R,DF)$ be a decomposition of $K$ and $s$ be a $\pi_0$ facet of $K$. Then the image of $s$ by the sum of the decomposition interval function 
is equal to $1$.

\end{sublemma}


\begin{sublemma}[HomologyFacet\_interval]
Let $(R,DF)$ be a decomposition of $K$ and $s$ be a homology facet of $K$. Then the decomposition interval defined by $s$ is equal to
$\{s\}$.

\end{sublemma}

\begin{sublemma}[Sum\_on\_DecompositionInterval\_HomologyFacet]
Let $(R,DF)$ be a decomposition of $K$ and $s$ be a homology facet of $K$. Then the image of $s$ by the sum of the decomposition interval function 
is equal to $(-1)^{card(s)-1}$.

\end{sublemma}

\begin{sublemma}[EulerPoincareCharacteristicDecomposable]
Let $(R,DF)$ be a decomposition of $K$, and suppose that $K$ is finite. Then the Euler-Poincaré characteristic of $K$ is equal to the
cardinality of the set of $\pi_0$ facets plus the sum over all homology facets of the function $s\mapsto (-1)^{card(s)-1}$.

\end{sublemma}

