\section{Finite weighted complex}

We fix a set $\alpha$ and a function $\mu:\alpha\to\R$.

\subsection{Generalities about $\mu$}

\begin{sublemma}[Exists\_LinearOrder\_antitone]
There exists a linear order on $\alpha$ that makes the function $\mu$ antitone. 

\end{sublemma}


\begin{sublemma}[Positivity\_condition]
If $\alpha$ is finite, then $\mu$ is summable and the sum of $\mu$ is nonnegative if and only if it is
nonnegative as a finite sum.

\end{sublemma}

From now on, we assume that $\mu$ is summable and the sum of $\mu$ is nonnegative.

\subsection{Positive sets}

\begin{subdefi}[IsPositiveSet]
Let $X$ be a finite subset of $\alpha$. We say that $X$ is a \emph{positive set} if $\sum_{a\in X}\mu(a)\geq 0$.

\end{subdefi}

\begin{subdefi}[AFLOPowerset\_positive]
Definition of the set of positive elements of AFLOPowerset: these are the $E$ such that every element of $E$ is a positive set.

\end{subdefi}

\begin{subdefi}[AFLOPowerset\_positive.PartialOrder]
Definition of the partial order on the set of positive elements of AFLOPowerset, by lifting the partial order given by inclusion.

\end{subdefi}

\begin{sublemma}[AFLOPartitions\_forget\_positive]
The set of positive elements of AFLOPowerset is included in AFLOPowerset.

\end{sublemma}



\begin{subdefi}[AFLOPartitions\_positive]
The set of positive AFLO partitions: these are AFLO partitions whose image by powersetToPreorder is a positive element of AFLOPowerset.

\end{subdefi}

\begin{subdefi}[AFLOPartitions\_positive.PartialOrder]
The partial order on positive AFLO partitions obtained by lifting the partial order on preorders.

\end{subdefi}

\begin{sublemma}[AFLOPartitions\_forget\_positive]
The set of positive AFLO partitions is a subset of the set of AFLO partitions.

\end{sublemma}

\begin{sublemma}[AFLOPowerset\_positive\_down\_closed]
The set of positive elements of AFLOPowerset is down-closed (in the set of all finite sets of subsets of $\alpha$).

\end{sublemma}


\subsection{The weighted complex}

\begin{subdefi}[WeightedComplex]
The weighted complex is the abstract simplicial complex on the powerset of $\alpha$ whose faces are nonempty positive elements of AFLOPowerset.

\end{subdefi}

\begin{sublemma}[FacesWeightedComplex]
Let $s$ be a finite set of subsets of $\alpha$. Then $s$ is a face of the weighted complex if and only it is a nonempty positive element of AFLOPowerset.
    
\end{sublemma}

\begin{sublemma}[WeightedComplex\_subcomplex]
The weighted complex is a subcomplex of the Coxeter complex.    

\end{sublemma}

\begin{sublemma}[AFLO\_positive\_powersetToPreorder]
If $E$ is a positive element of AFLOPowerset, then $powersetToPreorder(E)$ is a positive AFLO partition.

\end{sublemma}

\begin{sublemma}[AFLO\_positive\_preorderToPowerset]
If $s$ is a positive AFLO partition, then $preorderToPowerset(s)$ is a positive element of AFLOPowerset.

\end{sublemma}

\begin{subdefi}[WeightedComplextoPositivePartitions]
The isomorphism of ordered sets between the set of positive elements of AFLOPowerset and the dual of the set of positive AFLO partitions,
given by the maps $powersetToPreorder$ and $preorderToPowerset$.

\end{subdefi}

\begin{sublemma}[WeightedFaces\_powersetToPreordertoPower]
If $s$ is a face of the weighted complex, then
$s=preorderToPowerset(powersetToPreorder(s))$.

\end{sublemma}

\begin{sublemma}[AFLOPartitions\_positive\_IsUpperSet]
Positive AFLO partial form an upper set (in the set of all preorders on $\alpha$).

\end{sublemma}


\subsection{The restriction map}

\begin{sublemma}[AscentPartition\_respects\_AFLO\_positive]
If $r$ is a linear order on $\alpha$ and $s$ is a positive AFLO partition, then the ascent partition $AP(r,s)$ is a positive AFLO partition.

\end{sublemma}

\begin{subdefi}[restriction\_weighted]
If $r$ is a linear order on $\alpha$, we define the restriction map from the set of positive elements of AFLOPowerset to itself: it sends
$E$ to $preorderToPowerset(AP(r,powersetToPreorder(E)))$.
    
\end{subdefi}

\begin{sublemma}[restriction\_weighted\_is\_smaller]
If $r$ is a linear order on $\alpha$ and $E$ is a positive element of AFLOPowerset, then the image of $E$ by the restriction map is included
in $E$.

\end{sublemma}

\begin{subdefi}[R\_weighted]
The restriction map as a map from the set of facets of the weighted complex to the set of finite sets of subsets of $\alpha$.

\end{subdefi}

\subsection{The case of a finite set}

From now, we suppose that $\alpha$ is finite.

\begin{sublemma}[AFLOPartitions\_iff]
Let $p$ be a preorder on $\alpha$. Then $p$ is a positive AFLO partition if and only if it is total and every element of $preorderToPowerset(p)$ is
a positive set.

\end{sublemma}

\begin{sublemma}[AFLOPowerset\_positive\_iff]
Let $E$ be a finite set of subsets of $\alpha$. Then $E$ is a positive element of AFLOPowerset if and only if it is totally ordered by inclusion,
doesn't contain $\varnothing$ and $\alpha$ and all its elements are positive sets.

\end{sublemma}

\begin{sublemma}[WeightedComplex\_nonempty\_iff]
The weighted complex is nonempty if and only if $card(\alpha)\geq 2$.

\end{sublemma}

\begin{sublemma}[WeightedComplex\_all\_iff]
The weighted complex is equal to the Coxeter complex if and only if $card(\alpha)<2$ or $\mu(a)\geq 0$ for every element $a$ of $\alpha$.
\footnote{This should also be true for infinite $\alpha$.}

\end{sublemma}


\subsection{The distinguished facet map}

\begin{sublemma}[LinearOrder\_etc\_respects\_AFLO\_positive]
Let $r$ be a linear order on $\alpha$ such that $\mu$ is antitone, and let $s$ be a positive AFLO partition. Then the associated linear order
$LO(r,s)$ is a positive AFLO partition.

\end{sublemma}

\begin{subdefi}[distinguishedFacet\_weighted]
If $r$ is a linear order on $\alpha$ such that $\mu$ is antitone, we define the distinguished facet map from the set of positive elements of 
AFLOPowerset to itself: it sends $E$ to $preorderToPowerset(LO(r,powersetToPreorder(E)))$.
    
\end{subdefi}

\begin{sublemma}[distinguishedFacet\_weighted\_is\_smaller]
If $r$ is a linear order on $\alpha$ such that $\mu$ is antitone and $E$ is a positive element of AFLOPowerset, then the image of $E$ by the 
distinguished facet map contains $E$.

\end{sublemma}

\begin{sublemma}[distinguishedFacet\_weighted\_is\_facet\_CoxeterComplex]
If $r$ is a linear order on $\alpha$ such that $\mu$ is antitone and $E$ is a facet of the weighted complex, then the image of $E$ by
the distinguished facet map is a facet of the Coxeter complex.

\end{sublemma}

\begin{sublemma}[distinguishedFacet\_weighted\_is\_face\_WeightedComplex]
If $r$ is a linear order on $\alpha$ such that $\mu$ is antitone and $E$ is a facet of the weighted complex, then the image of $E$ by
the distinguished facet map is a face of the weighted complex.

\end{sublemma}

\begin{sublemma}[FacetWeightedComplex\_iff]
Let $s$ be a finite set of subsets of $\alpha$. Then $s$ is a facet of the weighed complex if and only if it is a facet of the Coxeter complex
and a face of the weighted complex.

\end{sublemma}

\begin{sublemma}[R\_comparison]
If $r$ is a linear order on $\alpha$ and $s$ is a facet of the weighted complex, then its image by the maps $R$ for the Coxeter complex and
the weighted complex are equal.

\end{sublemma}


\begin{subdefi}[DF\_weighted]
If $r$ is a linear order on $\alpha$ such that $\mu$ is antitone,
the distinguished facet map as a map from the set of faces of the weighted complex to the set of facets of the weighted complex.

\end{subdefi}


\begin{sublemma}[DF\_comparison]
If $r$ is a linear order on $\alpha$ such that $\mu$ is antitone and $s$ is a face of the weighted complex, then its image by the distinguished
facet maps for the Coxeter complex and the weighted complex are equal.

\end{sublemma}

\begin{sublemma}[R\_weighted\_eq\_empty\_iff]
Let $r$ be a linear order on $\alpha$ and $s$ be a facet of the weighed complex. Then $R(s)=\varnothing$ if and only if $powersetToPreorder(s)=r$.

\end{sublemma}

\begin{sublemma}[R\_weighted\_eq\_self\_iff]
Let $r$ be a linear order on $\alpha$ such that $\mu$ is antitone and $s$ be a facet of the weighed complex. Then $R(s)=s$ if and only if $powersetToPreorder(s)$ is 
the dual of $r$.

\end{sublemma}


\subsection{Facets of the weighted complex}

\begin{sublemma}[Fixed\_linear\_order\_in\_AFLO\_positive]
If $r$ is a linear order on $\alpha$ such that $\mu$ is antitone, then $r$ is a positive AFLO partition.

\end{sublemma}

\begin{sublemma}[Fixed\_linear\_order\_in\_WeightedComplex]
If $r$ is a linear order on $\alpha$ such that $\mu$ is antitone and if $card(\alpha)\geq 2$, then $preorderToPowerset(r)$ is a face of the weighted complex.

\end{sublemma}

\begin{sublemma}[Dual\_linear\_order\_in\_AFLO\_positive]
If $r$ is a linear order on $\alpha$ such that $\mu$ is antitone, then the dual $r$ is a positive AFLO partition if and only $\mu(a)\geq 0$ for every element
$a$ of $\alpha$.

\end{sublemma}

\begin{sublemma}[Dual\_linear\_order\_in\_WeightedComplex]
If $r$ is a linear order on $\alpha$ such that $\mu$ is antitone and if $card(\alpha)\geq 2$, then the image of the dual $r$ by $preorderToPowerset$ is a 
face of the weighted complex if and only $\mu(a)\geq 0$ for every element

\end{sublemma}


\subsection{Decomposability of the weighted complex}

\begin{sublemma}[WeightedComplex\_is\_decomposable]
If $r$ is a linear order on $\alpha$ such that $\mu$ is antitone, then the pair $(R,DF)$ is a decomposition of the weighted complex.

\end{sublemma}

\subsection{Some properties of the weighted complex}

\begin{sublemma}[WeightedComplex\_is\_finite]
The weighted complex is finite.

\end{sublemma}

\begin{sublemma}[WeightedComplex\_dimension\_facet]
If $s$ is a facet of the weighted complex, then $card(s)=card(\alpha)-1$.

\end{sublemma}

\begin{sublemma}[WeightedComplex\_is\_pure]
The weighted complex is pure.

\end{sublemma}

\begin{sublemma}[WeightedComplex\_Pi0Facet]
Let $r$ be a linear order on $\alpha$ such that $\mu$ is antitone and let $s$ be a facet of the weighted complex. Then $s$ is a $\pi_0$ facet if
and only if $powersetToPreorder(s)=r$ or $card(\alpha)=2$.

\end{sublemma}

\begin{sublemma}[WeightedComplex\_HomologyFacet]
Let $r$ be a linear order on $\alpha$ such that $\mu$ is antitone and let $s$ be a facet of the weighted complex. Then $s$ is a homology facet if
and only if $powersetToPreorder(s)$ is equal to the dual of $r$ and $card(\alpha)>2$.

\end{sublemma}


\subsection{Shellability of the weighted complex}

\begin{subdefi}[WeightedComplexFacets\_to\_LinearOrders]
A map from the set of facets of the weighted complex to the set of linear orders on $\alpha$ (given by $powersetToPreorder$).

\end{subdefi}

\begin{sublemma}[WeightedComplexFacets\_to\_LinearOrders\_injective]
The map of the previous definition is injective.

\end{sublemma}

\begin{subdefi}[WeakBruhatOrder\_facets\_WeightedComplex]
Let $r$ be a linear order on $\alpha$. We define the weak Bruhat order (relative to $r$) on the set of facets of the weighted complex by
lifting the weak Bruhat on the set of linear orders on $\alpha$.

\end{subdefi}

\begin{sublemma}[WeakBruhat\_compatible\_with\_DF\_weighted]
Let $r$ be a linear order on $\alpha$ such that $\mu$ is antitone. Then the weak Bruhat order on the set of facets of the weighted complex
is compatible with the map $DF$.

\end{sublemma}

\begin{sublemma}[WeightedComplexShelling]
Let $r$ be a linear order on $\alpha$ such that $\mu$ is antitone. Then every linear order on the set of facets of the weighted complex refining
the weak Bruhat order is a shelling order.

\end{sublemma}


\subsection{Euler-Poincaré characteristic of the weighted complex}

\begin{subdefi}[FacetWeightedComplexofLinearOrder]
Let $r$ be a linear order on $\alpha$ such that $\mu$ is antitone. If $card(\alpha)\geq 2$, this defines the facet of the weighted complex
corresponding to $preorderToPowerset(r)$.

\end{subdefi}

\begin{sublemma}[WeightedComplex\_of\_pair]
Let $a,b$ be elements of $\alpha$ such that $a\ne b$ and $\alpha$ is equal to $\{a,b\}$.  If $\mu(a)\geq 0$ and $\mu(b)<0$, then a finite set
of subsets of $\alpha$ is a face of the weighted complex if and only it is equal to $\{\{a\}\}$.

\end{sublemma}

\begin{sublemma}[preorderToPowerset\_of\_pair]
Let $r$ be a linear order on $\alpha$ and $a,b$ be elements of $\alpha$ such that $a<_r b$ and $\alpha$ is equal to $\{a,b\}$.  Then 
$preorderToPowerset(r)$ is equal to $\{\{a\}\}$.

\end{sublemma}

\begin{sublemma}[EulerPoincareCharacteristic\_WeightedComplex]
Suppose that $\alpha$ is nonempty. Then the Euler-Poincaré characteristic of the weighted complex is equal to $1+(-1)^{card(\alpha)}$ if
$\mu(a)\geq 0$ for every element $a$ of $\alpha$, and to $1$ otherwise. 

\end{sublemma}

