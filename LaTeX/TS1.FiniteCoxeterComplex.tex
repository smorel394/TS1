\section{Coxeter complex of a finite symmetric group}

We fix a set $\alpha$. The goal of this section is to define the Coxeter complex (for $\alpha$ arbitrary) and to prove that it is shellable.
This relies on the fact that the Coxeter complex is decomposable, so we will define restriction and distinguished facet maps that depend on
an auxiliary linear order on $\alpha$; the restriction map makes sense in general, but for the distinguished facet map we need to assume that
$\alpha$ is finite (since otherwise the Coxeter complex has no facets).

\subsection{Definition of the Coxeter complex}

For now, we don't assume that $\alpha$ is finite, so we work with almost finite linearly ordered (AFLO) partitions and
their sets of lower sets, AFLOPowerset.

\begin{sublemma}[AFLOPowerset\_down\_closed]
If $E$ is an element of $AFLOPowerset$, then so is every subset of $E$.

\end{sublemma}

\begin{subdefi}[CoxeterComplex]
The \emph{Coxeter complex} is the abstract simplicial complex on the powerset of $\alpha$ whose faces are the nonempty elements of $AFLOPowerset$.

\end{subdefi}

\begin{sublemma}[FacesCoxeterComplex]
Let $s$ be a finite set of subset of $\alpha$. Then $s$ is a face of the Coxeter complex if and only if $s$ is in AFLOPowerset and $s\ne\varnothing$.

\end{sublemma}

\begin{subdefi}[CoxeterComplextoPartitions]
An isomorphism of ordered sets between $AFLOPowerset$ and the dual of the set of AFLO partitions, given the functions $powersetToPreorder$ and $preorderToPowerset$.


\end{subdefi}

\begin{sublemma}[Faces\_powersetToPreordertoPowerset]
If $s$ is a face of the Coxeter complex, then $s=preorderToPowerset(powersetToPreorder(s))$.

\end{sublemma}

\begin{sublemma}[CoxeterComplex\_dimension\_face]
Let $s$ be an element of $AFLOPowerset$, and suppose that $\alpha$ is nonempty. Then the cardinality of $s$ is equal to the
cardinality of the antisymmetrization of $s$ minus $1$.

\end{sublemma}

\begin{sublemma}[twoStepPreorder\_AFLO]
For every element $a$ of $\alpha$, the two-step preorder defined by $a$ is an AFLO partition.

\end{sublemma}

\begin{sublemma}[twoStepPreorder\_in\_CoxeterComplex]
Let $a,b$ be elements of $\alpha$ such that $a\ne b$. Then the image by $preorderToPowerset$ of the two-step preorder defined by $a$
is a face of the Coxeter complex.

\end{sublemma}

\begin{sublemma}[AFLOPartitions\_IsUpperSet]
AFLO partitions form an upper set of the set of preorders of $\alpha$.

\end{sublemma}


\subsection{The restriction map}

\begin{sublemma}[AscentPartition\_respects\_AFLO]
Let $r$ be a linear order on $\alpha$ and $s$ be an AFLO partition. Then the ascent partition $AP(r,s)$ is an AFLO partition.

\end{sublemma}

\begin{subdefi}[restriction]
If $r$ is a linear order on $\alpha$ and $E$ is in $AFLOPowerset$, we define the image of $E$ by the restriction map as an element
of $AFLOPowerset$: we take $powersetToPreorder(E)$, apply the function $AP(r,\cdot)$ and then apply $preorderToPowerset$.

\end{subdefi}

\begin{sublemma}[restriction\_is\_smaller]
If $r$ is a linear order on $\alpha$ and $E$ is in $AFLOPowerset$, the image of $E$ by the restriction map is contained in $E$.

\end{sublemma}

\begin{subdefi}[R]
If $r$ is a linear order on $\alpha$, we define a map $R$ from the set of facets of the Coxeter complex to the set of finite sets of
subsets of $\alpha$ by restricting the restriction map that we just defined.

\end{subdefi}


\subsection{The case of a finite set}

From now, we assume that $\alpha$ is finite.

\begin{sublemma}[AFLOPartitions\_is\_everything]
Let $s$ be a preorder on $\alpha$. Then $s$ is an AFLO partition if and only if it is a linearly ordered partition (i.e. a total preorder).

\end{sublemma}

\begin{sublemma}[AFLOPowerset\_is\_everything]
Let $E$ be a finite set of subsets of $\alpha$. Then $E$ is in AFLOPowerset if and only if it is totally ordered by inclusion and does not contain
$\varnothing$ and $\alpha$.

\end{sublemma}

\begin{sublemma}[Facets\_are\_linear\_orders]
Let $s$ be face of the Coxeter complex. Then $s$ is a facet if and only $powersetToPreorder(s)$ is a linear order on $\alpha$.

\end{sublemma}

\begin{sublemma}[R\_eq\_empty\_iff]
Let $r$ be a linear order on $\alpha$ and $s$ be a facet of the Coxeter complex. Then $R(s)=\varnothing$ if and only if
$preorderToPowerset(s)=r$.

\end{sublemma}


\begin{sublemma}[R\_eq\_self\_iff]
Let $r$ be a linear order on $\alpha$ and $s$ be a facet of the Coxeter complex. Then $R(s)=s$ if and only if
$preorderToPowerset(s)$ is the dual of $r$.

\end{sublemma}


\subsection{The distinguished facet map}

\begin{sublemma}[LinearOrder\_etc\_respects\_AFLO]
Let $r$ be a linear order on $\alpha$ and $s$ be an AFLO partition. Then the associated linear order $LO(r,s)$ is an AFLO partition.

\end{sublemma}

\begin{subdefi}[distinguishedFacet]
If $r$ is a linear order on $\alpha$ and $E$ is in $AFLOPowerset$, we define the image of $E$ by the distinguished facet map as an element
of $AFLOPowerset$: we take $powersetToPreorder(E)$, apply the function $LO(r,\cdot)$ and then apply $preorderToPowerset$.

\end{subdefi}

\begin{sublemma}[distinguishedFacet\_is\_bigger]
If $r$ is a linear order on $\alpha$ and $E$ is in $AFLOPowerset$, the image of $E$ by the distinguished facet map contains in $E$.

\end{sublemma}

\begin{sublemma}[distinguishedFacet\_is\_facet]
If $r$ is a linear order on $\alpha$ and $E$ is a face of the Coxeter complex, the image of $E$ by the distinguished facet map is a facet
of the Coxeter complex.

\end{sublemma}

\begin{subdefi}[DF]
The map $DF$ is the distinguished facet map, seen as a map from the set of faces of the Coxeter complex to the set of facets of
the Coxeter complex.

\end{subdefi}



\subsection{Decomposability of the Coxeter complex}

\begin{sublemma}[CoxeterComplex\_is\_decomposable]
The pair $(R,DF)$ is a decomposition of the Coxeter complex.

\end{sublemma}


\subsection{Some properties of the Coxeter complex}

\begin{sublemma}[CoxeterComplex\_nonempty\_iff]
The Coxeter complex is nonempty if and only of $card(\alpha)$ is at least $2$.

\end{sublemma}

\begin{sublemma}[CoxeterComplex\_is\_finite]
The Coxeter complex is finite.

\end{sublemma}

\begin{sublemma}[NonemptyType\_of\_face\_CoxeterComplex]
If $s$ is a face of the Coxeter complex, then $\alpha$ is nonempty.
\footnote{Well duh... This is here for convenience.}

\end{sublemma}

\begin{sublemma}[CoxeterComplex\_dimension\_facet]
Let $s$ be a facet of the Coxeter complex. Then $card(s)=card(\alpha)-1$.

\end{sublemma}

\begin{sublemma}[CoxeterComplex\_is\_pure]
The Coxeter complex is pure.

\end{sublemma}

\begin{sublemma}[CoxeterComplex\_Pi0Facet]
Let $r$ be a linear order on $\alpha$ and $s$ be a face of the Coxeter complex. Then $s$ is a $\pi_0$ facet if and only if
$powersetToPreorder(s)=r$ or $card(\alpha)=2$.

\end{sublemma}

\begin{sublemma}[CoxeterComplex\_HomologyFacet]
Let $r$ be a linear order on $\alpha$ and $s$ be a face of the Coxeter complex. Then $s$ is a homology facet if and only if
$powersetToPreorder(s)$ is the dual of $r$ and $card(\alpha)>2$.

\end{sublemma}


\subsection{Shellability of the Coxeter complex}

\begin{subdefi}[CoxeterComplexFacetstoLinearOrders]
A map from the set of facets of the Coxeter complex to the set of linear orders on $\alpha$ (given by $powersetToPreorder$).

\end{subdefi}

\begin{sublemma}[CoxeterComplexFacetstoLinearOrders\_injective]
The map from the previous definition is injective.

\end{sublemma}

\begin{subdefi}[LinearOrderstoCoxeterComplexFacets]
If $card(\alpha)\geq 2$, a map from linear orders on $\alpha$ to facets of the Coxeter complex (given by $preorderToPowerset$).

\end{subdefi}

\begin{subdefi}[FacetsCoxeterComplextoLinearOrders]
If $card(\alpha)\geq 2$, an equivalence between facets of the Coxeter complex and linear orders on $\alpha$ (given by $powersetToPreorder$ and
$preorderToPowerset$).

\end{subdefi}

\begin{sublemma}[FacetsCoxeterComplex.card]
If $card(\alpha)\geq 2$, then the cardinality of the set of facets of the Coxeter complex is $(card \alpha)!$.

\end{sublemma}

This lemma is the only one that does not have a Lean proof, so we provide a proof in the case $card(\alpha)=2$ (which is the only one we need).

\begin{sublemma}[Elements\_pair]
If $a,b,c$ are elements of $\alpha$ such that $\alpha$ is equal to $\{a,b\}$, then $c=a$ or $c=b$.

\end{sublemma}

\begin{sublemma}[twoStepPreorder\_linear]
If $a,b$ are elements of $\alpha$ such that $\alpha$ is equal to $\{a,b\}$, then the two-step preorder associated to $a$ is a linear order.

\end{sublemma}

\begin{sublemma}[fintypePreorder]
The set of preorders on $\alpha$ is finite.

\end{sublemma}

\begin{sublemma}[CardLinearOrders\_pair]
If $card(\alpha)=2$, then the set of linear orders on $\alpha$ has cardinality $2$.

\end{sublemma}

\begin{sublemma}[FacetsCoxeterComplex.card\_pair]
If $card(\alpha)=2$, then the set of facets of the Coxeter complex has cardinality $2$.

\end{sublemma}

\begin{subdefi}[WeakBruhatOrder\_facets]
If $r$ is a linear order on $\alpha$, this defines the weak Bruhat order relative to $r$ on the set of facets of the Coxeter complex, by lifting the
weak Bruhat on linear orders.

\end{subdefi}

\begin{sublemma}[WeakBruhat\_compatible\_with\_DF]
Let $r$ be a linear order on $\alpha$. Then the weak Bruhat order on the facets of the Coxeter complex is compatible with the map $DF$.

\end{sublemma}

\begin{sublemma}
Let $r$ be a linear order on $\alpha$. Then any linear order on the set of facets of the Coxeter complex refining the weak Bruhat is a shelling order.

\end{sublemma}


\subsection{Euler-Poincaré characteristic of the Coxeter complex}

\begin{subdefi}[FacetofLinearOrder]
If $card(\alpha)>1$, the facet of the Coxeter complex associated to a linear order $r$ on $\alpha$.
\footnote{Didn't we already have this definition before ? Why repeat it ?}

\end{subdefi}

\begin{sublemma}[EulerPoincareCharacteristic\_CoxeterComplex]
If $\alpha$ is nonempty, then the Euler-Poincaré characteristic of the Coxeter complex is equal to $1+(-1)^{card(\alpha)}$.

\end{sublemma}

