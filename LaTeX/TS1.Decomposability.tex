\section{Decomposable abstract simplicial complexes}

We fix a set $\alpha$ and an abstract simplicial complex $K$ on $\alpha$.

\subsection{Definition of decomposability}

\begin{subdefi}[IsDecomposition]
Let $R$ be a function from the set of facets of $K$ to the set of finite subset of $\alpha$ and $DF$ be a function
from the set of faces of $K$ to the set of facets of $K$.
We say that $R$ and $DF$ define a decomposition of $K$ if:
\begin{itemize}
\item[(1)] For every facet $s$ of $K$, we have $R(s)\subset s$.
\item[(2)] For every face $s$ of $K$ and every facet $t$ of $K$, we have $R(t)\subset s\subset t$ if and only $DF(s)=t$.

\end{itemize}
\end{subdefi}

In what follows, we just say "$(R,DF)$ is a decomposition of $K$".

\begin{sublemma}[Decomposition\_DF\_bigger\_than\_source]
Let $(R,DF)$ be a decomposition of $K$.
Then, for every face $s$ of $K$, we have $s\subset DF(s)$.

\end{sublemma}

\begin{sublemma}[Decomposition\_is\_union]
Let $(R,DF)$ be a decomposition of $K$.
Then, for every face $s$ of $K$, there exists a facet $t$ of $K$ such that $R(t)\le s\le t$.

\end{sublemma}

\begin{sublemma}[Decomposition\_is\_disjoint]
Let $(R,DF)$ be a decomposition of $K$ and $s$ be a face of $K$.
If $t_1,t_2$ are facets of $K$ such that $R(t_1)\le s\le t_1$ and $R(t_2)\le s\le t_2$, then
$t_1=t_2$.

\end{sublemma}

\begin{sublemma}[Decomposition\_DF\_of\_a\_facet]
Let $(R,DF)$ be a decomposition of $K$ and $s$ be a facet of $K$. Then $DF(s)=s$.

\end{sublemma}

\begin{sublemma}[Decomposition\_image\_of\_R]
Let $(R,DF)$ be a decomposition of $K$ and $s$ be a facet of $K$. If $R(s)\ne\varnothing$, then
$R(s)$ is a face of $K$.

\end{sublemma}

\begin{sublemma}[Decomposition\_image\_of\_R']
Let $(R,DF)$ be a decomposition of $K$ and $s$ be a facet of $K$. Then $R(s)=\varnothing$ or
$R(s)$ is a face of $K$.

\end{sublemma}

\begin{sublemma}[Decomposition\_SF\_composed\_with\_R]
Let $(R,DF)$ be a decomposition of $K$ and $s$ be a facet of $K$. If $R(s)\ne\varnothing$, then $s=DF(R(s))$.

\end{sublemma}

\begin{sublemma}[Decomposition\_R\_determines\_DF]
Let $(R,DF_1)$ and $(R,DF_2)$ be decompositions of $K$. Then $DF_1=DF_2$.

\end{sublemma}


\subsection{Intervals of a decomposition}

\begin{subdefi}[Interval]
If $s$ is empty of a face of $K$ and $t$ is a face of $K$, we define $Interval(s,t)$ as the finite set of faces $u$ of $K$
such that $s\subset u\subset t$.

\end{subdefi}

\begin{subdefi}[DecompositionInterval]
Let $(R,DF)$ be a decomposition of $K$ and $s$ be a facet of $K$. The corresponding \emph{decomposition interval}
is $Interval(R(s),s)$.

\end{subdefi}

\begin{subdefi}[DecompositionInterval\_def]
Let $(R,DF)$ be a decomposition of $K$ and $s$ be a facet of $K$. If $t$ is a face of $K$, then $t$ is in the decomposition
interval of $s$ if and only $R(s)\subset t\subset t$.

\end{subdefi}

\begin{subdefi}[DecompositionInterval\_eq]
Let $(R,DF)$ be a decomposition of $K$ and $s$ be a facet of $K$. If $t$ is a face of $K$, then $t$ is in the decomposition
interval of $s$ if and only $DF(t)=s$.

\end{subdefi}

\begin{sublemma}[Decomposition\_DF\_determines\_R\_intervals]
Let $(R_1,DF)$ and $(R_2,DF)$ be decompositions of $K$. Then, for every facet $s$ of $K$, we have
$Interval(R_1(s),s)=Interval(R_2(s),s)$.

\end{sublemma}


\subsection{Compatible partial orders on facets}

\begin{subdefi}[CompatibleOrder]
Let $DF$ be a map from the set of faces of $K$ to the set of facets of $K$ and $r$ be a partial order on the set of
facets of $K$. We say that $r$ is \emph{compatible} with $DF$ if, for every face $s$ of $K$ and every facet $t$ of $K$,
if $s\le t$, then $DF(s)\le_r t$.

\end{subdefi}

\begin{sublemma}[OldFacesCompatibleOrder]
Let $DF$ be a map from the set of faces of $K$ to the set of facets of $K$ and $r$ be a partial order on the set of
facets of $K$ that is compatible with $DF$. Let $s$ be a facet of $K$ and $t$ be a face of $K$ such that $t\le s$ and
$t\le DF(t)$. Then $t$ is not in the complex of old faces of $s$ (relative to $r$) if and only if $DF(t)=s$.

\end{sublemma}

\begin{sublemma}[OldFacesDecomposition]
Let $(R,DF)$ be decomposition of $K$ and $r$ be a partial order on the set of
facets of $K$ that is compatible with $DF$. Let $s$ be a facet of $K$ and $t$ be a face of $K$ such that $t\le s$.
Then $t$ is not in the complex of old faces of $s$ (relative to $r$) if and only if $t$ is in the decomposition interval
corresponding to $s$.

\end{sublemma}

\begin{sublemma}[OldFacesDecomposition']
Let $(R,DF)$ be decomposition of $K$ and $r$ be a partial order on the set of
facets of $K$ that is compatible with $DF$. Let $s$ be a facet of $K$ and $t$ be a face of $K$ such that $t\le s$.
Then $t$ is not in the complex of old faces of $s$ (relative to $r$) if and only if $R(s)\subset t$.

\end{sublemma}

\begin{sublemma}[OldFacesDecomposition\_faces]
Let $(R,DF)$ be decomposition of $K$ and $r$ be a partial order on the set of
facets of $K$ that is compatible with $DF$. Let $s$ be a facet of $K$ and $t$ be a face of $K$ such that $t\le s$.
Then $t$ is in the complex of old faces of $s$ (relative to $r$) if and only if $R(s)$ is not included in $t$.

\end{sublemma}

\begin{sublemma}[OldFacesDecomposition\_empty\_iff]
Let $(R,DF)$ be decomposition of $K$ and $r$ be a partial order on the set of
facets of $K$ that is compatible with $DF$. Let $s$ be a facet of $K$. Then the complex of old faces of $s$ relative
to $r$ is empty if and only the decomposition interval of $s$ is equal to the half-infinite interval $]\leftarrow,s]$.

\end{sublemma}

\begin{sublemma}[OldFacesDecompositionDimensionFacets]
Let $(R,DF)$ be decomposition of $K$ and $r$ be a partial order on the set of
facets of $K$ that is compatible with $DF$. Let $s$ be a facet of $K$ and $t$ a facet of the complex of old faces of $s$ relative
to $r$. Then the cardinality of $t$ is equality to the cardinality of $s$ minus $1$.

\end{sublemma}

\begin{sublemma}[OldFacesDecompositionIsPure]
Let $(R,DF)$ be decomposition of $K$ and $r$ be a partial order on the set of
facets of $K$ that is compatible with $DF$. Let $s$ be a facet of $K$. Then the complex of old faces of $s$ relative
to $r$ is pure.

\end{sublemma}

\begin{sublemma}[OldFacesDecompositionDimension]
Let $(R,DF)$ be decomposition of $K$ and $r$ be a partial order on the set of
facets of $K$ that is compatible with $DF$. Let $s$ be a facet of $K$. Then the complex of old faces of $s$ relative
to $r$ is of dimension $card(s)-2$.

\end{sublemma}


\subsection{$\pi_0$ and homology facets}

\begin{subdefi}[IsPi0Facet]
Let $(R,DF)$ be decomposition of $K$ and $s$ be a facet of $K$. We say that $s$ is a \emph{$\pi_0$ facet} if the decomposition
interval of $s$ is equal to the half-infinite interval $]\leftarrow,s]$.

\end{subdefi}

\begin{subdefi}[IsHomologyFacet]
Let $(R,DF)$ be decomposition of $K$ and $s$ be a facet of $K$. We say that $s$ is a \emph{homology facet} if it is not a $\pi_0$ facet
and if the decomposition interval of $s$ is equal to the singleton $\{s\}$.

\end{subdefi}

\begin{sublemma}[Vertex\_IsPi0Facet]
Let $(R,DF)$ be decomposition of $K$ and $s$ be a facet of $K$. If the cardinality of $s$ is equal to $1$, then $s$ is a $\pi_0$ facet.

\end{sublemma}

\begin{sublemma}[IsPi0Facet\_iff]
Let $(R,DF)$ be decomposition of $K$ and $s$ be a facet of $K$. Then $s$ is a $\pi_0$ facet if and only if $R(s)$ is empty or the
cardinality of $s$ is equal to $1$.

\end{sublemma}

\begin{sublemma}[IsHomologyFacet\_iff]
Let $(R,DF)$ be decomposition of $K$ and $s$ be a facet of $K$. Then $s$ is a homology facet if and only if $R(s)=s$ is empty and the
cardinality of $s$ is $>1$.

\end{sublemma}










