\documentclass[10pt,leqno]{scrartcl}

% \usepackage[francais]{babel} %% what is the difference with french, or
                 %% frenchb?

\usepackage{a4wide} %% maybe not so good for the
            %% non-europeans. Should we use the geometry package?

%% in order to use accented characters.

%\usepackage{isolatin1}

\usepackage{theorem}

\usepackage{amsmath}

%% for other fonts like \mathbb and \mathfrak; amssymb contains
%% amsfonts.... So that gives more choice for the typists.

\usepackage{amssymb}

%% for simple rectangular diagrams

\usepackage{amscd}

%% for the more complicated diagrams one can use the XY-pic package.

\usepackage[all]{xy}

%% Why not use the times fonts

\usepackage{times}

%% for a maybe nicer script font

\usepackage{mathrsfs}

%% We want fancy headings

\usepackage{fancyheadings}
\usepackage{hyperref}

%% This comes from the LaTeX companion, pages 92-93.

\newcommand{\clearemptydoublepage}
{\newpage{\pagestyle{empty}\cleardoublepage}}

%% We want the chapter numbers in large Roman numbers.

% \renewcommand{\thechapter}{\Roman{chapter}}

%% we want the numbering of the sections and subsections to be as in
%% the book, i.e., without the chapter number in front of them

\renewcommand{\thesection}{\arabic{section}}
\renewcommand{\thesubsection}{\thesection.\arabic{subsection}}
\renewcommand{\thesubsubsection}{\thesubsection.\arabic{subsubsection}}
\renewcommand{\theparagraph}{\thesubsubsection.\arabic{paragraph}}
\renewcommand{\theequation}{\arabic{equation}}

%% We define the theorem environments

%% First those that have their text underlined (becomes slanted).

%%\theorembodyfont{\sl}
\newtheorem{prop}[subsection]{Proposition}
\newtheorem{lemma}[subsection]{Lemma}
\newtheorem{thm}[subsection]{Theorem}
\newtheorem{cor}[subsection]{Corollary}

\newtheorem{sublemma}[subsubsection]{Lemma}
\newtheorem{subthm}[subsubsection]{Theorem}
\newtheorem{subprop}[subsubsection]{Proposition}
\newtheorem{subcor}[subsubsection]{Corollary}


%% And those that do not have their text underlined.

\theorembodyfont{\rmfamily}

\newtheorem{defi}[subsection]{Definition}
\newtheorem{rmk}[subsection]{Remark}
\newtheorem{exemple}[subsection]{Exemple}
\newtheorem{remarques}[subsection]{Remarques}
\newtheorem{fait}[subsection]{Fait}

\newtheorem{subdefi}[subsubsection]{Definition}
\newtheorem{subrmk}[subsubsection]{Remark}
\newtheorem{subex}[subsubsection]{Example}
\newtheorem{subremarques}[subsubsection]{Remarques}
\newtheorem{subnotation}[subsubsection]{Notation}
\newtheorem{subfact}[subsubsection]{Fact}

%% And a *-form:

\newtheorem{remarquesstar}{Remarques}
\renewcommand{\theremarquesstar}{}
\newtheorem{exemplesstar}{Exemples}
\renewcommand{\theexemplesstar}{}
\newtheorem{remarquestar}{Remarque}
\renewcommand{\theremarquestar}{}
\newtheorem{anything}{}


%% Preuves

\newenvironment{proof}{\vspace{0.3cm}\noindent
\emph{Proof.}}{\hfill $\square$ \vspace{0.3cm}}
\newenvironment{proofarg}[1]{\vspace{0.3cm}\noindent
\emph{Proof of \nobreakspace#1.}}{\hfill
$\square$ \vspace{0.3cm}}

%% for the equations: this will not work for every chapter since the
%% numbering depends heavily on the chapter

\numberwithin{equation}{subsubsection}

%% redefine \subsection so that the text following it starts on the
%% same line.

% \makeatletter
% \renewcommand{\subsection}
% {\@startsection {subsection}{2}{\z@ }{-3.25ex\@plus -1ex \@minus
% -.2ex} {-1.5ex}{\normalfont \normalsize \bfseries }} \makeatother


%% redefine \subsubsection so that the text following it starts on the
%% same line.

% \makeatletter
% \renewcommand{\subsubsection}
% {\@startsection {subsubsection}{3}{\z@ }{-3.25ex\@plus -1ex
% \@minus -.2ex} {-1.5ex}{\normalfont \normalsize \bfseries }}
% \makeatother

%% redefine \paragraphsection so that the text following it starts on the
%% same line.

% \makeatletter
% \renewcommand{\paragraph}
% {\@startsection {paragraph}{4}{\z@ }{-3.25ex\@plus -1ex \@minus
% -.2ex} {-1.5ex}{\normalfont \normalsize \bfseries }} \makeatother

%% Some commands for the typesetting choices....

%% The choice for the script font. Some underlined math symbols become
%% script. For example: \mathcal{O} for structure sheaves.

\renewcommand{\mathcal}{\mathscr}

%% The choice for the black board bold font.

\renewcommand{\Bbb}{\mathbb}

%% The choice for the gothic font. These are handwritten in the typed
%% text.

\newcommand{\goth}{\mathfrak}

%% make the old font selection commands from LaTeX 2.09 work as the
%% ones we want, but only in math mode.

\renewcommand{\rm}{\mathrm}
\renewcommand{\it}{\mathit}
\renewcommand{\bf}{\mathbf}
\renewcommand{\cal}{\mathcal}

%% Macros. See AMSLaTex user's guide for explanations.

%% Ensembles

\newcommand{\C}{\mathbb{C}}
\newcommand{\Nat}{\mathbb{N}}
\newcommand{\R}{\mathbb{R}}
\newcommand{\Q}{\mathbb{Q}}
\newcommand{\Z}{\mathbb{Z}}
\newcommand{\Fi}{\mathbb{F}}
\newcommand{\Af}{\mathbb{A}_f}
\newcommand{\Aff}{\mathbb{A}}
\newcommand{\Proj}{\mathbb{P}}

%% Cohomologie

\newcommand{\Hp}{{}^p\mathrm{H}}
\renewcommand{\H}{\mathrm{H}}
\newcommand{\HH}{\mathcal{H}}
\newcommand{\HyH}{\mathbb{H}}
\newcommand{\Hc}{\check{\mathrm{H}}}

%% Hom

\DeclareMathOperator{\Hom}{Hom}
\DeclareMathOperator{\Ext}{Ext}
\DeclareMathOperator{\Homf}{\underline{Hom}}
\DeclareMathOperator{\Extf}{\underline{Ext}}

%% Groupes

\newcommand{\SD}{\mathbb{S}}
\newcommand{\GL}{\bf {GL}}
%%\newcommand{\SL}{\bf SL}
\newcommand{\GSp}{\bf GSp}
\newcommand{\Sp}{\bf Sp}
\newcommand{\GU}{\bf GU}
\newcommand{\SU}{\bf SU}
\newcommand{\Gr}{\mathbb{G}}
\newcommand{\Pa}{\bf P}
\newcommand{\QP}{\bf Q}
\newcommand{\RP}{\bf R}
\newcommand{\B}{\bf B}
\newcommand{\N}{\bf N}
\newcommand{\Le}{\bf L}
\newcommand{\Se}{\bf S}
\newcommand{\Ar}{\mathrm{A}}
\newcommand{\Hr}{\mathrm{H}}

%% Faisceaux

\newcommand{\F}{\mathcal{F}}
\newcommand{\Gf}{\mathcal{G}}
\newcommand{\Of}{\mathcal{O}}
\newcommand{\Hf}{\mathcal{H}}
\newcommand{\Kf}{\mathcal{K}}
\newcommand{\Mf}{\mathcal{M}}
\newcommand{\Bf}{\mathcal{B}}
\newcommand{\Uf}{\mathcal{U}}
\newcommand{\Cf}{\mathcal{C}}
\newcommand{\Pf}{\mathcal{P}}

%% Divers

\newcommand{\Ab}{\mathbf{Ab}}  %% category of abelian groups
\newcommand{\adj}{\mathrm{adj}} 
\newcommand{\aff}{\mathrm{aff}} 
\newcommand{\as}{\underline{a}}
\DeclareMathOperator{\can}{can}
\DeclareMathOperator{\card}{card} 
\DeclareMathOperator{\Chow}{Chow}
\DeclareMathOperator{\codim}{codim}
\DeclareMathOperator{\Coker}{Coker}
\DeclareMathOperator{\D}{D} %% cat�gorie d�riv�e
\DeclareMathOperator{\DF}{DF} %% cat�gorie d�riv�e filtr�e
\DeclareMathOperator{\DM}{DM} %% motifs mixtes
\newcommand{\DP}{{}^{w}\D}
\newcommand{\Dp}{{}^{p}\D}
\DeclareMathOperator{\diag}{diag}
\DeclareMathOperator{\E}{E} %% spectral sequence (I.2.6)
\newcommand\equv{\Longleftrightarrow}
\newcommand{\et}{\mathrm{\acute{e}t}} %% plus accent
\newcommand{\fl}{\longrightarrow}
\newcommand{\ra}{\longrightarrow}
\newcommand{\Ra}{\Longrightarrow}
\def\flnom#1{\stackrel{#1}{\fl}}
\newcommand{\fle}{\longmapsto}
\DeclareMathOperator{\Frac}{Frac}
\DeclareMathOperator{\Gra}{Gr} %% graded
\DeclareMathOperator{\Gal}{Gal} %% for Galois group
\newcommand{\id}{\mathrm{id}} %% morphisme identit� 
\renewcommand{\Im}{\mathrm{Im}} %% replaces old \Im (Imaginary part
                 %% of complex number)
\DeclareMathOperator{\Ind}{Ind}
\newcommand{\iso}{\stackrel{\sim}{\fl}}
\DeclareMathOperator{\Ker}{Ker}
\newcommand{\lin}{\ell} 
\newcommand{\Lf}{\mathcal{L}} %% syst�me local
\newcommand{\Ltimes}{\otimes^L}
\newcommand{\LS}{\mathrm{LS}}
\newcommand{\PLS}{\mathrm{PLS}}
\newcommand{\SL}{\mathrm{SL}}
\newcommand{\M}{\mathcal{M}} %% cat�gorie de faisceaux pervers
\DeclareMathOperator{\Ob}{\mathrm{Ob}}
\newcommand{\op}{\mathrm{op}} 
\newcommand\oQ{\overline{\Q}}
\DeclareMathOperator{\Perv}{Perv}
\DeclareMathOperator{\Pro}{Pro}
\DeclareMathOperator{\PSh}{PSh}
\DeclareMathOperator{\DR}{R}  %% for right-derivation of functors
\DeclareMathOperator{\DL}{L} %% for left-derivation of functors
\DeclareMathOperator{\Pre}{Pre}
\newcommand\quash[1]{}
\newcommand{\real}{\mathrm{real}} %% foncteur r�alisation
\newcommand{\red}{\mathrm{r\acute{e}d}} %% for reduced closed subscheme
\newcommand{\Sf}{\mathcal{S}}
\newcommand{\Sch}{\mathbf{Sch}}
\newcommand{\Sets}{\mathrm{Sets}} 
\DeclareMathOperator{\Sh}{Sh}
\newcommand{\ssi}{si et seulement si}
\newcommand{\sous}{\setminus}
\DeclareMathOperator{\Spec}{Spec}
\DeclareMathOperator{\supp}{supp} %% for the support of a sheaf
\DeclareMathOperator{\Sym}{Sym} %% puissances sym�triques
\newcommand{\tame}{\mathrm{t}}
\newcommand{\TR}{\mathfrak{TR}}
\newcommand{\til}{\widetilde}
\newcommand{\U}{\mathcal{U}}
\newcommand{\ungras}{1\!\!\mkern -1mu1}
\newcommand{\var}{\mathrm{var}}
\newcommand{\X}{\mathcal{X}}
\newcommand{\Y}{\mathcal{Y}}
\newcommand{\Zar}{\mathrm{Zar}} 

%% For comments to myself.

% \newcommand\footnotevar[1]{\footnote{#1}}
\newcommand\footnotevar[1]{}

%% In order to prevent line breaks in formulas.

\relpenalty=10000 \binoppenalty=10000


%% In order to avoid too many over hboxes and so on for the moment.

\sloppy


%% The following is an attempt to make the table of contents better,
%% as the Roman numbers were too wide.

% \makeatletter

% \renewcommand*\l@chapter[2]{%
%   \ifnum \c@tocdepth >\m@ne
%     \addpenalty{-\@highpenalty}%
%     \vskip 1.0em \@plus\p@
%    \setlength\@tempdima{2.5em}%
%    \begingroup
%      \parindent \z@ \rightskip \@pnumwidth
%      \parfillskip -\@pnumwidth
%      \leavevmode \bfseries
%      \advance\leftskip\@tempdima
%      \hskip -\leftskip
%      #1\nobreak\hfil \nobreak\hb@xt@\@pnumwidth{\hss #2}\par
%      \penalty\@highpenalty
%    \endgroup
%  \fi}
%
% \renewcommand{\l@section}
% {\@dottedtocline {1}{2.5em}{2.3em}}

% \makeatother


%% We make the space between paragraphs bigger (note: it is
%% temporarily changed back in the table of contents)

\setlength{\parskip}{0.5\baselineskip}


\begin{document}

This is an explanation of what happens in the Lean files leading up to TS1.lean. The goal is to define the "weighted complex" (a subcomplex of the Coxeter complex of the
symmetric group), to prove that it is shellable and to calculate its Euler-Poincaré characteristic, and then to deduce a formula for the alternating sum on totally partitions
of a finite set $\alpha$ satisfying a certain positivity condition. (The final formula is easy enough to prove by induction on the cardinality of $\alpha$, so it is mainly
an excuse to play with some interesting abstract simplicial complexes.)


The sections correspond to separate Lean files, the subsections to the
internal organizations of these files.

\section{General preorder stuff}

\subsection{General stuff}

This contains general lemmas about preorder on a type $\alpha$. If $s$ is a preorder on $\alpha$, we write $\simeq_s$ for the antisymmetrization relation of $s$
(i.e. $a\simeq_s b$ if and only if $a\le_s b$ and $b\le_s a$). This is an equivalence relation, and the quotient of $\alpha$ by this equivalence relation is
called the antisymmetrization; it carries a partial order induced by $s$.

\begin{sublemma}[TotalPreorder\_trichotomy]
If $s$ is a total preorder on $\alpha$, then, for all $a,b\in\alpha$, we have $a <_s b$, $b<_s a$ or $a\simeq_s b$.

\end{sublemma}

\begin{sublemma}[LinearPreorder\_trichotomy]
If $s$ is a linear preorder on $\alpha$ (i.e. a preorder that happens to be a linear order), then, for all $a,b\in\alpha$, we have $a <_s b$, $b<_s a$ or $a=b$.
    
\end{sublemma}

\begin{sublemma}[TotalPreorder\_lt\_iff\_not\_le]
If $s$ is a total preorder on $\alpha$ and $a,b\in\alpha$, then $\neg(a \le_s b)$ if and onlyb if $b<_s a$.

\end{sublemma}


\subsection{SuccOrders and antisymmetrization}

A SuccOrder structure on $s$ is the data of a "reasonable" successor function.

\begin{subdefi}[SuccOrdertoAntisymmetrization]
If hsucc is a SuccOrder on $\alpha$ for the preorder $s$, then we get a SuccOrder on the antisymmetrization of $\alpha$ (relative to $s$) by
taking as successor function the function sending $x$ to the image of the successor of a lift of $x$.

\end{subdefi}

\begin{subdefi}[SuccOrderofAntisymmetrization]
If $s$ is a preorder on $\alpha$ and hsucc is a SuccOrder on the antisymmetrization of $\alpha$ (for the canonical partial order), then we get
a SuccOrder on $\alpha$ for $s$ by sending $a$ to any lift of the successor of the image of $a$ in the antisymmetrization.

\end{subdefi}


\subsection{Essentially locally finite preorders}

\begin{subdefi}[EssentiallyLocallyFinitePreorder]
We say that a preorder is \emph{essentially locally finite} if its antisymmetrization partial order is locally finite (i.e. closed intervals are finite).

\end{subdefi}

\begin{subdefi}[EssentiallyLocallyFinite\_ofLocallyFinite]
Any structure of locally finite preorder on $s$ defines a structure of essentially locally finite preorder.

\end{subdefi}

\begin{subdefi}[TotalELFP\_SuccOrder]
Any structure of essentially locally finite preorder on a total preorder $s$ defines a SuccOrder structure.

\end{subdefi}

\begin{sublemma}[ELFP\_is\_locally\_WellFounded]
If $s$ is an essentially locally finite preorder, then it is well-founded on any closed interval.

\end{sublemma}

\subsection{Partial order on preorders}

\begin{subdefi}[instPreorder.le \& Preorder.PartialOrder]
We define a partial order on preorders by saying that $s \le t$ if only, for all $a,b\in\alpha$, $a \le_s b$ implies $a\le_t b$.

\end{subdefi}


\subsection{The trivial preorder}

\begin{subdefi}[trivialPreorder]
The trivial preorder on $\alpha$ is the preorder with graph $\alpha\times\alpha$.

\end{subdefi}

\begin{subdefi}[trivialPreorder\_is\_total]
The trivial preorder on $\alpha$ is total.

\end{subdefi}

\begin{sublemma}[trivialPreorder\_is\_greatest]
Every preorder is smaller than or equal to the trivial preorder.

\end{sublemma}

\begin{sublemma}[nontrivial\_preorder\_iff\_exists\_not\_le]
A preorder $s$ is different from the trivial preorder if and only if there exist $a,b\in\alpha$ such that $\neg(a \le_s b)$.

\end{sublemma}

\subsection{Partial order on preorders and antisymmetrization}

\begin{sublemma}[AntisymmRel\_monotone]
If $s,t$ are preorders on $\alpha$ such that $s \le t$, then the graph of $\simeq_s$ is contained in the graph of $\simeq_t$.

\end{sublemma}

\begin{subdefi}[AntisymmetrizationtoAntisymmetrization]
Let $s,t$ be preorders such that $s\le t$. We have a map from the antisymmetrization of $s$ to that of $t$ sending $x$ to the image 
in the antisymmetrization of $t$ of any lift of $x$.

\end{subdefi}

\begin{sublemma}[AntisymmetrizationtoAntisymmetrization\_lift]
If $s,t$ are preorders on $\alpha$ such that $s \le t$ and $a\in\alpha$, then the class of $a$ in the antisymmetrization of $t$ is the
image of its class in the antisymmetrization of $s$.

\end{sublemma}

\begin{sublemma}[AntisymmetrizationtoAntisymmetrization\_monotone]
If $s,t$ are preorders on $\alpha$ such that $s \le t$, then the map from the antisymmetrization of $s$ to the antisymmetrization
of $t$ is monotone.

\end{sublemma}

\begin{sublemma}[AntisymmetrizationtoAntisymmetrization\_image\_interval]
If $r,s$ are preorders on $\alpha$ such that $r\le s$ and $r$ is total, and if $a,b\in\alpha$ are such that $a\le_r b$, the the image
of the interval $[\overline{a},\overline{b}]$ in the antisymmetrization of $r$ is the corresponding interval in the
antisymmetrization of $s$.

\end{sublemma}


\subsection{Upper sets for the partial order on preorders}

\begin{sublemma}[Total\_IsUpperSet]
Total preorders form on upper set.

\end{sublemma}

\begin{subdefi}[TotalELPF\_IsUpperSet]
If $r$ is a total essentially locally finite preorder, then it defines a structure of essentially locally finite preorder on any $s$ such that
$r \le s$.

\end{subdefi}


\subsection{Noetherian preordered sets}

\begin{subdefi}[IsNoetherianPoset]
A preorder $s$ on $\alpha$ is called \emph{Noetherian} if the relation $>_s$ is well-founded.

\end{subdefi}

\subsection{Maximal nonproper order ideals}

We fix a preorder on $\alpha$.

\begin{subdefi}[Order.Ideal.IsMaximalNonProper]
An order ideal is called \emph{maximal nonproper} if it is maximal among all order ideals of $\alpha$.

\end{subdefi}

This definition is only interesting if $\alpha$ itself is not an order ideal, i.e. if the preorder on $\alpha$ is not directed.

\begin{sublemma}[OrderIdeals\_inductive\_set]
Order ideals of $\alpha$ form an inductive set (i.e. any nonempty chain has an upper bound).

\end{sublemma}

\begin{sublemma}[Order.Ideal.contained\_in\_maximal\_nonproper]
Any order ideal of $\alpha$ is contained in a maximal nonproper order ideal.

\end{sublemma}

\begin{sublemma}[Order.Ideal.generated\_by\_maximal\_element]
If $I$ is an order ideal of $\alpha$ and if $a$ is a maximal element of $I$, then $I$ is generated by $a$.

\end{sublemma}

\begin{sublemma}[Order.PFilter.generated\_by\_minimal\_element]
If $F$ is an order filter of $\alpha$ and if $a$ is a minimal element of $F$, then $F$ is generated by $a$.

\end{sublemma}

\begin{sublemma}[Noetherian\_iff\_every\_ideal\_is\_principal\_aux \& Noetherian\_iff\_every\_ideal\_is\_principal]
The preorder on $\alpha$ is Noetherian if and only every order ideal is principal (= generated by one element).

\end{sublemma}


\subsection{Map from $\alpha$ to its order ideals}

We fix a partial order on $\alpha$.

\begin{subdefi}[Elements\_to\_Ideal]
We have an order embedding from $\alpha$ to the set of its order ideals (ordered by inclusion) sending $a$ to the
ideal generated by $a$.

\end{subdefi}

\subsection{Locally finite partial order on finsets of $\alpha$}

\begin{sublemma}[FinsetIic\_is\_finite]
If $s$ is a finset of $\alpha$, then the half-infinite interval $]\leftarrow,s]$ is finite.

\end{sublemma}

\begin{sublemma}[FinsetIcc\_is\_finite]
If $s$ and $t$ are finsets of $\alpha$, then the closed interval $[s,t]$ is finite.

\end{sublemma}

\begin{subdefi}[FinsetLFB]
A structure of locally finite order with smallest element on the set of finsets of $\alpha$.

\end{subdefi}

\begin{subdefi}[FacePosetLF]
A structure of locally finite order with smallest element on the set of finsets of $\alpha$.

\end{subdefi}

\subsection{Two-step preorders}

\begin{subdefi}[twoStepPreorder]
If $a$ is an element of $\alpha$, we define a preorder twoStepPreorder(a) that makes $a$ strictly smaller than every other element
and all other elements of $\alpha$ equal.

\end{subdefi}

\begin{sublemma}[twoStepPreorder\_smallest]
If $a$ is an element of $\alpha$, then it is the smallest element for twoStepPreorder(a).

\end{sublemma}

\begin{sublemma}[twoStepPreorder\_greatest]
If $a,b$ are elements of $\alpha$ such that $a\ne b$, then any element of $\alpha$ is smaller than or equal to
$b$ for twoStepPreorder(a).

\end{sublemma}

\begin{sublemma}[twoStepPreorder\_IsTotal]
If $a$ is an element of $\alpha$, then twoStepPreorder(a) is a total preorder.

\end{sublemma}

\begin{sublemma}[twoStepPreorder\_nontrivial]
If $a,b$ are elements of $\alpha$ such that $a\ne b$, then twoStepPreorder(a) is nontrivial.

\end{sublemma}

\begin{subdefi}[twoStepPreorder\_singleton\_toAntisymmetrization]
For $a$ an element of $\alpha$, a map from $\{a\}$ to the antisymmetrization of $\alpha$ for twoStepPreorder(a).

\end{subdefi}

\begin{subdefi}[twoStepPreorder\_nonsingleton\_toAntisymmetrization]
If $a,b$ are elements of $\alpha$, a map from $\{a\}\cup\ast$ to the antisymmetrization of $\alpha$ for twoStepPreorder(a).

\end{subdefi}

\begin{sublemma}[twoStepPreorder\_singleton\_toAntisymmetrization\_surjective]
If $a$ is an element of $\alpha$ such that every element of $\alpha$ is equal to $a$, then the map from
$\{a\}$ to the antisymmetrization of $\alpha$ for twoStepPreorder(a) is surjective.

\end{sublemma}

\begin{sublemma}[twoStepPreorder\_nonsingleton\_toAntisymmetrization\_surjective]
If $a,b$ are elements of $\alpha$ such that $a \ne b$, then the 
map from $\{a\}\cup\ast$ to the antisymmetrization of $\alpha$ for twoStepPreorder(a) is surjective.

\end{sublemma}

\begin{sublemma}[twoStepPreorder\_Antisymmetrization\_finite]
If $a$ is an element of $\alpha$, then the antisymmetrization of $\alpha$ for twoStepPreorder(a) is finite.

\end{sublemma}

\begin{sublemma}[twoStepPreorder\_Antisymmetrization.card]
If $a,b$ are elements of $\alpha$ such that $a \ne b$, then the 
antisymmetrization of $\alpha$ for twoStepPreorder(a) has cardinality $2$.

\end{sublemma}

\subsection{A linear order on any type}

\begin{subdefi}[ArbitraryLinearOrder]
A definition of a linear order on $\alpha$ (by embedding $\alpha$ into the class of cardinals and lifting the linear order
on cardinals).

\end{subdefi}



\section{Abstract simplicial complexes}

Let $\alpha$ be a set.

\subsection{Generalities}

\begin{subdefi}[AbstractSimplicialComplex]
An \emph{abstract simplicial complex} $K$ on $\alpha$ is a set of nonempty finite subsets of $\alpha$ (called \emph{faces} of $K$) such that, if
$s$ is face of $K$ and $t$ is a nonempty subset of $s$, then $t$ is a face of $K$.
\end{subdefi}

\begin{subdefi}[of\_erase]
If we have a set $S$ of finite subsets of $\alpha$ that is a lower set for inclusion, then we get an abstract simplicial complex whose faces
are the nonempty elements of $S$.

\end{subdefi}

\begin{subdefi}[of\_subcomplex]
If $K$ is an abstract simplicial complex and $S$ is a set of faces of $K$ that is a lower set for inclusion, then we get an abstract simplicial with
set of faces $S$.

\end{subdefi}

\begin{sublemma}[face\_card\_nonzero]
If $K$ is an abstract simplicial complex and $s$ is a face of $K$, then the cardinality of $s$ is a nonzero natural number.

\end{sublemma}

\subsection{Vertices}

\begin{subdefi}[vertices]
Let $K$ be an abstract simplicial complex. Vertices of $K$ are elements $a$ of $\alpha$ such that $\{a\}$ is a face of $K$.

\end{subdefi}

\begin{sublemma}[mem\_vertices]
An element $a$ of $\alpha$ is a vertex of $K$ if and only if $\{a\}$ is a face of $K$.

\end{sublemma}

\begin{sublemma}[vertices\_eq]
The set of vertices of $K$ is the union of the faces of $K$.

\end{sublemma}

\begin{sublemma}[mem\_vertices\_iff]
An element $a$ of $\alpha$ is a vertex of $K$ if and only if there exists a face $s$ of $K$ such that $a\in s$.

\end{sublemma}

\begin{sublemma}[face\_subset\_vertices]
Every face of $K$ is contained in the set of vertices of $K$.

\end{sublemma}


\subsection{Facets}

\begin{subdefi}[facets]
A \emph{facet} of $K$ is a maximal face.

\end{subdefi}

\begin{sublemma}[mem\_facets\_iff]
A finite set of $\alpha$ is a facet of $K$ if and only if it is a face of $K$ and it is maximal among faces of $K$.

\end{sublemma}

\begin{sublemma}[facets\_subset]
Every facet of $K$ is a face of $K$.

\end{sublemma}


\subsection{Partial order on abstract simplicial complexes}

\begin{subdefi}
We define a partial order on the set of abstract simplicial complexes on $\alpha$ by saying that $K$ is less than or equal to $L$
if and only if the set of faces of $K$ is contained in the set of faces of $L$.

\end{subdefi}

\begin{subdefi}[Inf]
If $K$ and $L$ are two abstract simplicial complexes, then $Inf(K,L)$ is the abstract simplicial complexes whose set of faces is
the intersection of the set of faces of $K$ and the set of faces of $L$.

\end{subdefi}

\begin{subdefi}[Sup]
If $K$ and $L$ are two abstract simplicial complexes, then $Sup(K,L)$ is the abstract simplicial complexes whose set of faces is
the union of the set of faces of $K$ and the set of faces of $L$.

\end{subdefi}

\begin{subdefi}[DistribLattice]
A structure of distributive lattice on the set of abstract simplicial complexes on $\alpha$, whose max and min are given by $Sup$ and $Inf$.

\end{subdefi}

\begin{subdefi}[Top]
We define an abstract simplicial complex $Top$ such that every finite subset of $\alpha$ is a face of $Top$.

\end{subdefi}

\begin{subdefi}[Bot]
We define an abstract simplicial complex $Bot$ whose set of faces of empty.

\end{subdefi}

\begin{subdefi}[OrderBot]
A smallest element of the set of abstract simplicial complexes on $\alpha$ given by $Bot$.

\end{subdefi}

\begin{subdefi}[OrderTop]
A biggest element of the set of abstract simplicial complexes on $\alpha$ given by $Top$.

\end{subdefi}

\begin{subdefi}[SupSet]
If $s$ is a set of abstract simplicial complexes on $\alpha$, an abstract simplicial complex $SupSet(s)$ whose set of faces is the union
of the sets of faces of the elements of $s$.

\end{subdefi}

\begin{subdefi}[InfSet]
If $s$ is a set of abstract simplicial complexes on $\alpha$, an abstract simplicial complex $InfSet(s)$ whose set of faces is the intersection
of the sets of faces of the elements of $s$.

\end{subdefi}

\begin{subdefi}[CompleteLattice]
A structure of complete lattice on the set of abstract simplicial complexes on $\alpha$, with supremum and infimum functions given by $SupSet$ and $InfSet$.

\end{subdefi}

\begin{subdefi}[CompleteDistribLattice]
A structure of complete distributive lattice on the set of abstract simplicial complexes on $\alpha$.

\end{subdefi}


\subsection{Finite complexes}

\begin{subdefi}[FiniteComplex]
An abstract simplicial complex is called \emph{finite} if its set of faces is finite.

\end{subdefi}

\begin{sublemma}[Finite\_IsLowerSet]
Every complex smaller than or equal to a finite complex is also finite.

\end{sublemma}

\begin{sublemma}[FiniteComplex\_has\_finite\_facets]
If $K$ is a finite abstract simplicial complex, then its set of facets is finite.

\end{sublemma}


\subsection{Pure complexes and dimension}

\begin{subdefi}[dimension]
Let $K$ be an abstract simplicial complex. The \emph{dimension} of $K$ is the supremum (taken in $\Nat\cup\{\infty\}$) of on the set of faces of $K$ of the function
$s\mapsto card(s)-1$.

\end{subdefi}

\begin{subdefi}[Pure]
Let $K$ be an abstract simplicial complex. We say that $K$ is \emph{pure} if, for every facet $s$ of $K$, the dimension of $K$ is equal to $card(s) -1$.

\end{subdefi}


\subsection{Skeleton and link}

\begin{subdefi}[skeleton]
Let $K$ be an abstract simplicial complex and $d\in\Nat$. The \emph{$d$-skeleton} of $K$ is the abstract simplicial complexes whose faces are
all faces of $K$ of cardinality $\leq d+1$.

\end{subdefi}

\begin{subdefi}[link]
Let $K$ be an abstract simplicial complex and $s$ be a finite subset of $\alpha$. The \emph{link} of $s$ in $K$ is the abstract simplicial complex
whose faces are all the faces $t$ of $K$ that are disjoint from $s$ and such that $s\cup t$ is a face of $K$.

\end{subdefi}


\subsection{Simplicial maps}

\begin{subdefi}[SimplicialMap]
Let $K$ and $L$ be abstract simplicial complexes on the sets $\alpha$ and $\beta$ respectively. A \emph{simplicial map} from $K$ to $L$ is a map
$f$ from $\alpha$ to $\beta$ such that, for every face $s$ of $K$, the finite set $f(s)$ is a face of $L$.

\end{subdefi}

\begin{subdefi}[toFaceMap]
Given a simplicial map $f$ from $K$ to $L$, we get a map $f.toFaceMap$ from the set of faces of $K$ to the set of faces of $L$.

\end{subdefi}

\begin{subdefi}[comp]
Definition of the composition of simplicial maps: it is given by composing the maps on the underlying sets.

\end{subdefi}

\begin{subdefi}[id]
If $K$ is an abstract simplicial complex on $\alpha$, the identity of $K$ is the simplicial map given by the identity of $\alpha$.

\end{subdefi}

\begin{subdefi}[AbstractSimplicialComplexEquiv]
Let $K$ and $L$ be abstract simplicial complexes on the sets $\alpha$ and $\beta$ respectively. A \emph{simplicial equivalence} from $K$ to $L$
is the data of a simplicial map $f:\alpha\to\beta$ from $K$ to $L$ and a simplicial map $g:\beta\to\alpha$ from $L$ to $K$ such that, for
every face $s$ of $K$, we have $g(f(s))=s$ and, for every face $t$ of $L$, we have $f(g(t))=t$.

\end{subdefi}


\subsection{Subcomplex generated}

\begin{subdefi}[SubcomplexGenerated]
Let $K$ be a abstract simplicial complex and $F$ be a set of finite sets of $\alpha$. The \emph{subcomplex of $K$ generated by $F$} is the abstract
simplicial complex whose faces are the faces of $K$ that are contained in an element of $F$.

\end{subdefi}

\begin{sublemma}[SubcomplexGenerated\_mem]
Let $K$ be a abstract simplicial complex and $F$ be a set of finite sets of $\alpha$.
A finite subset $s$ of $\alpha$ is a face of the subcomplex of $K$ generated by $F$ if and only if $s$ is a face of $K$ and there exists $t\in F$ such that
$s \subset t$.

\end{sublemma}

\subsection{Boundary of a simplex}

\begin{subdefi}[Boundary]
Let $K$ be an abstract simplicial complex and $s$ be a face of $K$. The \emph{boundary} of $s$ is the abstract simplicial complex whose faces
are nonempty proper subsets of $s$.\footnote{Hey, we don't need $K$ here !}

\end{subdefi}

\begin{sublemma}[Boundary\_mem]
Let $K$ be an abstract simplicial complex and $s$ be a face of $K$. A finite subset $t$ of $\alpha$ is a face of the boundary of $s$ if and only
if $t$ is a face of $K$, $t$ is included in $s$ and $t \ne s$.

\end{sublemma}

\begin{sublemma}[BoundaryFinite]
Let $K$ be an abstract simplicial complex and $s$ be a face of $K$. Then the boundary of $s$ is a finite complex.

\end{sublemma}
\section{The face poset of an abstract simplicial complex}

We fix a set $\alpha$ and an abstract simplicial complex $K$ on $\alpha$.

\subsection{Definitions}

\begin{subdefi}[FacePoset] 
The \emph{face poset} of $K$ is the set of faces of $K$, partially ordered by inclusion.

\end{subdefi}

\begin{sublemma}[FacePosetIic\_is\_finite]
For every face $s$ of $K$< the half-infinite interval $]\leftarrow,s]$ is finite.

\end{sublemma}

\begin{sublemma}[FacePosetIcc\_is\_finite]
For all faces $s$ and $t$, the closed interval $[s,t]$ is finite.

\end{sublemma}

\begin{subdefi}[FacePosetLFB]
A structure of locally finite order with smallest element on the face poset of $K$.

\end{subdefi}

\begin{subdefi}[FacePosetLF]
A structure of locally finite order on the face poset of $K$.

\end{subdefi}

\subsection{Filters of the face poset}

\begin{sublemma}[FacePoset.Filters]
Every filter of the face poset if principal.

\end{sublemma}


\subsection{Ideals of the face poset}

\begin{subdefi}[SupportIdeal]
If $I$ is an ideal of the face poset, its \emph{support} is the union of all its members.

\end{subdefi}

\begin{subdefi}[IdealFromSet]
If $S$ a subset of $\alpha$, the subset of the face poset that it defines is the set of faces contained in $S$.

\end{subdefi}

\begin{sublemma}[SupportIdeal\_def]
Let $I$ be an ideal of the face poset. An element $a$ of $\alpha$ is in the support of $I$ if and only if there exists
a face $s$ of $K$ such that $s\in I$ and $a\in s$.

\end{sublemma}

\begin{sublemma}SupportIdeal\_eq
Let $I$ be an ideal of the face poset. An element $a$ of $\alpha$ is in the support of $I$ if and only if the face
$\{a\}$ of $K$ is an element of $I$.

\end{sublemma}

\begin{sublemma}[SupportIdeal\_contains\_faces]
Let $I$ be an ideal of the face poset. If $s$ is a face of $K$ such that $s\in I$, then $s$ is contained in the support of $I$.

\end{sublemma}

\begin{sublemma}[SupportIdeal\_monotone]
If $I,J$ are ideals of the face poset of $K$ such that $I\subset J$, then the support of $I$ is contained in the support of $J$.

\end{sublemma}

\begin{sublemma}[SupportIdeal\_nonempty]
If $I$ is an ideal of the face poset of $K$, then the support of $I$ is nonempty. (Note that ideals are nonempty by definition.)

\end{sublemma}

\begin{sublemma}[Finset\_SupportIdeal\_aux and Finset\_SupportIdeal]
Let $I$ be an ideal of the face poset of $K$. Then any nonempty finite subset of the support of $I$ is a face of $K$.

\end{sublemma}

\begin{sublemma}[MemIdeal\_iff\_subset\_SupportIdeal]
Let $I$ be an ideal of the face poset of $K$ and $s$ be a face of $K$. Then $s\in I$ if and only if $s$ is contained in the
support of $I$.

\end{sublemma}

\begin{sublemma}[SupportIdeal\_principalIdeal]
Let $s$ be a face of $K$. The support of the principal ideal generated by $s$ is equal to $s$.

\end{sublemma}

\begin{sublemma}[SupportIdeal\_top]
Let $I$ be an ideal of the face poset of $K$. If $I$ is equal to the set of all faces of $K$, then the support of $I$
is equal to the set of vertices of $K$.

\end{sublemma}

\begin{sublemma}[IdealFromSet\_only\_depends\_on\_vertices]
Let $S$ be a subset of $\alpha$. Then $S$ and its intersection with the set of vertices of $K$ define the same subset
of the face poset of $K$.

\end{sublemma}

\begin{sublemma}[IdealFromSet.IsLowerSet]
Let $S$ be a subset of $\alpha$. The subset of the face poset of $K$ that it defines is a lower set.

\end{sublemma}

\begin{sublemma}[IdealFromSupport]
Let $I$ be an ideal of the face poset of $K$. Then $I$ is equal to the subset defined by its support.

\end{sublemma}

\begin{sublemma}[IdealFromSet\_DirectedOn\_iff\_aux \& IdealFromSet\_DirectedOn\_iff]
Let $S$ be a subset of $\alpha$ contained in the set of vertices of $K$. Then the subset of faces defined by $S$ is
directed if and only every nonempty finite subset of $S$ is a face of $K$.

\end{sublemma}

\begin{sublemma}[PrincipalIdeal\_iff]
An ideal of the face poset of $K$ is principal if and only if its support is finite.

\end{sublemma}

\begin{sublemma}[Subideal\_of\_Principal\_is\_Principal]
If $I,J$ are ideals of the face poset of $K$ such that $I\subset J$ and $J$ is principal, then $I$ is principal.

\end{sublemma}

\begin{sublemma}[AllIdealsPrincipal\_iff\_AllMaximalNonProperIdealsPrincipal]
The following are equivalent:
\begin{enumerate}
\item Every ideal of the face poset of $K$ is principal.
\item Every nonproper maximal ideal of the face poset of $K$ is principal.


\end{enumerate}
\end{sublemma}

\begin{sublemma}[Facet\_iff\_principal\_ideal\_maximal]
Let $s$ be a face of $K$. Then $s$ is a facet if and only if the principal ideal that it generates is nonproper maximal.

\end{sublemma}

\begin{sublemma}[Top\_ideal\_iff\_simplex]
The following are equivalent:
\begin{enumerate}
\item The set of all faces of $K$ is an ideal.
\item Every nonempty finite subset of the set of vertices of $K$ is a face of $K$.

\end{enumerate}
\end{sublemma}


\subsection{Noetherianity of the face poset}

\begin{sublemma}[Noetherian\_implies\_every\_face\_contained\_in\_facet]
If the face poset of $K$ is Noetherian, then every face of $K$ is contained in a facet.

\end{sublemma}

\begin{sublemma}[Noetherian\_nonempty\_implies\_facets\_exist]
If the face poset of $K$ is Noetherian and $K$ is nonempty, then $K$ has a facet. 

\end{sublemma}

\begin{sublemma}[Finite\_dimensional\_implies\_Noetherian]
If the dimension of $K$ is finite, then the face poset of $K$ is Noetherian.

\end{sublemma}

\begin{sublemma}[Finite\_implies\_Noetherian]
If $K$ is a finite complex, then its face poset is Noetherian.

\end{sublemma}

\begin{sublemma}[Finite\_implies\_finite\_dimensional]
If $K$ is a finite complex, then its dimension is finite.

\end{sublemma}

\begin{sublemma}[Dimension\_of\_Noetherian\_pure]
If the face poset of $K$ is Noetherian and all facets of $K$ have the same cardinality, then $K$ is pure.

\end{sublemma}

\section{Orders on the facets of a complex}

We fix a set $\alpha$ and an abstract simplicial complex $K$ on $\alpha$.

\subsection{The complex of old faces}

\begin{subdefi}[OldFaces]
Let $r$ be a partial order on the set of facets of $K$ and $s$ be a facet of $K$. We consider the abstract simplicial subcomplex
$L$ of $K$ generated by all facets $t$ such that $t<_r s$. The complex of \emph{old faces} defined by $r$ and $s$
is the abstract simplicial subcomplex of $L$ generated by $s$.

\end{subdefi}

\begin{sublemma}[OldFaces\_mem]
Let $r$ be a partial order on the set of facets of $K$ and $s$ be a facet of $K$. A finite subset $t$ of $\alpha$ is a face of
the complex of old faces if and only $t$ is a face of $K$, $t\subset s$ and there exists a facet $u$ of $K$ such that $u<_r s$ and
$t\subset u$.

\end{sublemma}

\begin{sublemma}[OldFaces\_included\_in\_boundary]
Let $r$ be a partial order on the set of facets of $K$ and $s$ be a facet of $K$. The complex of old faces is included in the
boundary of $s$.

\end{sublemma}

\begin{sublemma}[OldFacesFinite]
Let $r$ be a partial order on the set of facets of $K$ and $s$ be a facet of $K$. The complex of old faces is finite.

\end{sublemma}

\begin{sublemma}[OldFacesNonempty\_implies\_not\_vertex]
Let $r$ be a partial order on the set of facets of $K$ and $s$ be a facet of $K$. If the complex of old faces is nonempty,
then $s$ has cardinality $\geq 2$.

\end{sublemma}


\section{Going between total preorders and their sets of of lower sets}

We fix a set $\alpha$.

\subsection{From a preorder to the powerset}

\begin{subdefi}[preorderToPowerset]
If $s$ is a preorder on $\alpha$, then $preorderToPowerset(s)$ is the set of lower sets of $s$ that are nonempty and
not equal $\alpha$ itsself.

\end{subdefi}

\begin{sublemma}[preorderToPowerset\_TrivialPreorder\_is\_empty]
If $s$ is the trivial preorder on $\alpha$, then $preorderToPowerset(s)$ is empty.

\end{sublemma}

\begin{sublemma}[preorderToPowerset\_TwoStepPreorder]
If $s$ is the two-step preorder associated to $a$ and if $\alpha$ has an element different from $a$, then
$preorderToPowerset(s)$ is equal to $\{\{a\}\}$.

\end{sublemma}

\begin{sublemma}[preorderToPowerset\_total\_is\_total]
If $s$ is a total preorder, then $preorderToPowerset(s)$ is totally ordered by inclusion.

\end{sublemma}

\begin{sublemma}[preorderToPowerset\_antitone]
The function $preorderToPowerset$ is antitone.

\end{sublemma}


\subsection{From a set of subsets to a preorder}

\begin{subdefi}[powersetToPreorder]
If $E$ is a set of subsets of $\alpha$, we define a preorder $s=powersetToPreorder(E)$ on $\alpha$ by
setting $a\le_s b$ if and only every element of $E$ that contains $b$ also contains $a$.

\end{subdefi}

\begin{sublemma}[powersetToPreorder\_antitone]
The function $powersetToPreorder$ is antitone.

\end{sublemma}

\begin{sublemma}[powersetToPreorder\_total\_is\_total]
Let $E$ be a set of subsets of $\alpha$. If $E$ is totally ordered by inclusion, then $powersetToPreorder(E)$ is a
total preorder.

\end{sublemma}

\subsection{Going in both directions}

\begin{sublemma}[preorderToPowersetToPreorder]
For every preorder $s$ on $\alpha$, we have $s=powersetToPreorder(preorderToPowerset(s))$.

\end{sublemma}

\begin{sublemma}[preorderToPowerset\_injective]
The function $preorderToPowerset$ is injective.

\end{sublemma}

\begin{sublemma}[preorderToPowerset\_is\_empty\_iff\_TrivialPreorder]
Let $s$ be a preorder on $\alpha$. Then $preorderToPowerset(s)$ is empty if and only if $s$ is the trivial preorder on $\alpha$.

\end{sublemma}

\begin{sublemma}[powersetToPreorderToPowerset]
Let $E$ be a set of subsets of $\alpha$. Every element of $E$ that is a proper subset of $\alpha$ is in
$preorderToPowerset(powersetToPreorder(E))$.

\end{sublemma}


\subsection{The case of essentially locally finite preorders}

Under some conditions we have $E=preorderToPowerset(powersetToPreorder(E))$ in the last lemma (up to $\varnothing$ and $\alpha$). 
The condition I first wanted to use is "$s$ is total, locally (i.e. in each closed interval) the relation $>_s$ is well-founded 
and there is a successor function for $s$." But this is actually equivalent to the fact that, in every closed interval, the relations
$<_s$ and $>_s$ are well-founded. Indeed:
\begin{itemize}
\item[(1)]If the latter condition is true, then we get a successor function in the following way: Let $a$ be an element of $\alpha$. 
If $a$ is maximal, we set $succ(a) = a$. If not, then there exists $b$ such that $a <_s b$. We set $succ(a)$ to be a minimal element 
of the set $\{c | a <_s c \mbox{ and } c \le_s b\}$, which is a nonempty subset of the closed interval $[a,b]$. 
\item[(2)] In the other direction, let $a$ and $b$ be elements of $\alpha$ such that $a \le_s b$. We want to show that $<_s$ is 
well-founded on the closed interval $[a,b]$, so let $S$ be a nonempty subset of $[a,b]$. If there exists $c$ in $S$ such that $c \le_s a$, 
then this $c$ is minimal. Otherwise, the set $M$ of elements of $[a,b]$ that are stricly smaller than every element of $S$ is nonempty 
(because is contains $a$). Let $b$ be a maximal element of $M$, and let $c = succ(b)$. We have $b <_s c$ because $b$ is not maximal in
$\alpha$, so $c \not\in  M$, so there exists $d$ in $S$ such that $d \le_s c$. I claim that $d$ is a minimal element of $S$. Indeed, let 
$e$ be an element of $S$. If $e <_s d$, then $e <_s c$, so $e \le_s b$ (by the properties of the successor function), and this is a 
contradiction; so $\neg(e <_s d)$.
\end{itemize}

On the other hand, if $s$ is a total preorder such that $<_s$ and $>_s$ are well-founded, then the antisymmetrization of $s$ is finite. 
Indeed, going to the antisymmetrization, we may assume that $s$ is a partial order, hence a linear order, hence a well-order whose dual
is also a well-order. As every well-order is isomorphic to an ordinal, we may assume that $\alpha$ is an ordinal (and $s$ is its canonical
order). Then if $\alpha$ were infinite, it would contain the ordinal $\omega$ as an initial segment, and $\omega$ has no greatest element, 
so the dual of $s$ would not be a well-order.
  
So in conclusion, the conclusion I wanted to impose is equivalent to the fact that the antisymmetrization partial order of $s$ is
locally finite, which is the "essentially locally finite" condition.

\begin{sublemma}[TotalELFP\_LowerSet\_is\_principal]
If $s$ is total and essentially locally finite, then every element of $preorderToPowerset(s)$ is a half-infinite ideal
$]\leftarrow,a]$.

\end{sublemma}

\begin{sublemma}[TotalELFP\_powersetToPreorderToPowerset]
Let $E$ be a set of subsets of $\alpha$. If $powersetToPreorder(E)$ is total and essentially locally finite, then
$preorderToPowerset(powersetToPreorder(E))$ is included in $E$.

\end{sublemma}

\begin{sublemma}[preorderToPowersets\_down\_closed]
Let $s$ be a preorder on $\alpha$. If $s$ is total and essentially locally finite, then, for every subset $E$ of $preorderToPowerset(s)$,
we have $E=preorderToPowerset(powersetToPreorder(E))$.

\end{sublemma}


\subsection{Relation between the set of lower sets and the antisymmetrization}

\begin{subdefi}[Preorder\_nonmaximal]
We define the set of nonmaximal elements of a preorder $s$.

\end{subdefi}

\begin{subdefi}[Antisymmetrization\_nonmaximal]
We define the set of nonmaximal elements of the antisymmetrization partial order.

\end{subdefi}

\begin{sublemma}[Antisymmetrization\_nonmaximal\_prop1]
Let $s$ be a preorder on $\alpha$. If $s$ has no maximal element, then the set of nonmaximal elements of its antisymmetrization
partial order is equal to its antisymmetrization.

\end{sublemma}

\begin{sublemma}[Antisymmetrization\_nonmaximal\_prop2]
Let $s$ be a preorder on $\alpha$. If $s$ has a maximal element $s$, then the set of nonmaximal elements of its antisymmetrization
partial order is equal to the set of elements of its antisymmetrization that are not equal to the image of $a$.

\end{sublemma}

\begin{sublemma}[FiniteAntisymmetrization\_exists\_maximal]
Let $s$ be a preorder on $\alpha$. If the antisymmetrization of $s$ is finite and $\alpha$ is nonempty, then $s$ has a maximal element.

\end{sublemma}

\begin{sublemma}[FiniteAntisymmetrization\_nonmaximal]
Let $s$ be a preorder on $\alpha$. If the antisymmetrization of $s$ is finite and $\alpha$ is nonempty, then the set of nonmaximal elements of 
the antisymmetrization partial order of $s$ is equal to the set of elements of its antisymmetrization that are not equal to the image of a
an arbitrarily chosen maximal element of $s$.

\end{sublemma}

\begin{sublemma}[Antisymmetrization\_to\_powerset]
Let $s$ be a preorder on $\alpha$. We have a map from the antisymmetrization of $s$ to the powerset of $\alpha$ sending $x$ to the set
of $a$ in $\alpha$ whose image is $\le x$.

\end{sublemma}

\begin{sublemma}[Antisymmetrization\_to\_powerset\_in\_PreorderToPowerset]
Let $s$ be a total preorder on $\alpha$ and $x$ be a nonmaximal element of the antisymmetrization of $s$. Then the subset of $\alpha$
defined by $x$ is in $preorderToPowerset(s)$.

\end{sublemma}

\begin{sublemma}[Antisymmetrization\_to\_powerset\_preserves\_order]
Let $s$ be a total preorder on $\alpha$ and $x,y$ be a elements of the antisymmetrization of $s$. Then $x\le y$ if and only if
the set defined by $x$ is contained in the set defined by $y$.

\end{sublemma}

\begin{sublemma}[Antisymmetrization\_to\_powerset\_injective]
Let $s$ be preorder on $\alpha$. The map from the antisymmetrization of $s$ to the powerset of $\alpha$ is injective.

\end{sublemma}

\begin{sublemma}[Nonempty\_of\_mem\_PreorderToPowerset]
Let $s$ be a preorder on $\alpha$. If $preorderToPowerset(s)$ has an element, then $\alpha$ is nonempty.

\end{sublemma}

\begin{sublemma}[Antisymmetrization\_to\_powerset\_surjective]
Let $s$ be a total essentially locally finite preorder on $\alpha$. For every element $X$ of $preorderToPowerset(s)$, there exists
a nonmaximal element of the antisymmetrization of $s$ defining $X$.

\end{sublemma}

\begin{subdefi}[Equiv\_Antisymmetrization\_nonmaximal\_to\_PreorderToPowerset]
Let $s$ be a total essentially locally finite preorder on $\alpha$. We define an equivalence between the set of nonmaximal elements
of the antisymmetrization of $s$ and $preorderToPowerset(s)$.

\end{subdefi}

\begin{subdefi}[OrderIso\_Antisymmetrization\_minus\_greatest\_to\_PreorderToPowerset]
Under the same hypotheses, we upgrade the previous equivalence to an order isomorphism.

\end{subdefi}
\section{Linearly ordered partitions}

We fix a set $\alpha$.

\begin{subdefi}[dual]
If $r$ is a linear order on $\alpha$, then $dual(r)$ is the dual linear order.

\end{subdefi}


\subsection{The partial order on linearly ordered partitions}

\begin{subdefi}[LinearOrderedPartitions]
A linearly ordered partitions on $\alpha$ is a total preorder on $\alpha$.

\end{subdefi}

\begin{subdefi}[LinearOrderedPartitions.PartialOrder]
We restrict the partial order on preorders to a partial order on linearly ordered partitions.

\end{subdefi}

\begin{sublemma}[trivialPreorder\_is\_greatest\_partition]
The trivial preorder is the greatest element of the set of linearly ordered partitions.

\end{sublemma}

\begin{sublemma}[linearOrder\_is\_minimal\_partition]
A linear order is minimal in the set of linearly ordered partitions.

\end{sublemma}


\subsection{Cooking up a linear order from a total preorder}

\begin{subdefi}[LinearOrder\_of\_total\_preorder\_and\_linear\_order\_aux]
Let $r$ be a linear order on $\alpha$ and $s$ be a preorder on $\alpha$. We define a relation
$LO(r,s)$ on $\alpha$ by saying that $a\le_{LO(r,s)}b$ if and only if $a<_s b$, or $a\simeq_s b$ and
$a\le_r b$. 

\end{subdefi}

The idea is that we want to define a linear order smaller than or equal to $s$, and we use $r$ to order the elements
of $\alpha$ that are equivalent for $s$.

\begin{sublemma}[LinearOrder\_of\_total\_preorder\_and\_linear\_order\_aux \& LinearOrder\_of\_total\_preorder\_and\_linear\_order\_refl \& LinearOrder\_of\_total\_preorder\_and\_linear\_order\_trans
\& LinearOrder\_of\_total\_preorder\_and\_linear\_order\_antisymm \& LinearOrder\_of\_total\_preorder\_and\_linear\_order\_total]
Let $r$ be a linear order on $\alpha$ and $s$ be a total preorder on $\alpha$. Then $LO(r,s)$ is a linear order on $\alpha$.

\end{sublemma}

\begin{subdefi}[LinearOrder\_of\_total\_preorder\_and\_linear\_order]
The relation $LO(r,s)$ as a preorder.

\end{subdefi}

\begin{sublemma}[LinearOrder\_of\_total\_preorder\_and\_linear\_order\_is\_total]
The preorder $LO(r,s)$ is total if $s$ is total.

\end{sublemma}

\begin{sublemma}[LinearOrder\_of\_total\_preorder\_and\_linear\_order\_is\_linear]
The preorder $LO(r,s)$ is a linear order if $s$ is total.

\end{sublemma}

\begin{sublemma}[LinearOrder\_of\_total\_preorder\_and\_linear\_order\_is\_smaller]
The preorder $LO(r,s)$ is smaller than or equal to $s$.

\end{sublemma}

\begin{sublemma}[LinearOrder\_vs\_fixed\_LinearOrder]
If $s$ is total and $a,b$ are elements of $\alpha$ such that $a\simeq_s b$, then
$a\le_r b$ if and only if $a\le_{LO(r,s)}b$.

\end{sublemma}

\begin{sublemma}[LinearOrder\_of\_total\_preorder\_and\_linear\_order\_lt]
Suppose that $s$ is total, and let $a,b$ be elements of $\alpha$. If $a<_{LO(r,s)}b$, then $a<_s b$.

\end{sublemma}

\begin{sublemma}[LinearOrder\_of\_linear\_order\_and\_linear\_order\_is\_self]
If $s$ is a linear order, then $LO(r,s)=s$.

\end{sublemma}

\begin{sublemma}[minimal\_partition\_is\_linear\_order]
A minimal linearly ordered partition is a linear order.

\end{sublemma}

We finish this subsection with some lemmas about the principal lower sets of $LO(r,s)$.

\begin{sublemma}[LowerSet\_LinearOrder\_etc\_is\_disjoint\_union]
Let $a$ be an element of $\alpha$ and $X$ be the half-infinite interval $]\leftarrow,a]$ for the preorder $LO(r,s)$. Then
$X$ is the disjoint union of the half-infinite interval $]\leftarrow,a[$ for the preorder $s$ and of the set
$\{b| b\le_r a\mbox{ and }a\simeq_s b\}$.

\end{sublemma}

\begin{sublemma}[LowerSet\_LinearOrder\_etc\_is\_difference]
Let $a$ be an element of $\alpha$ and $X$ be the half-infinite interval $]\leftarrow,a]$ for the preorder $LO(r,s)$. Then
$X$ is the difference of the half-infinite interval $]\leftarrow,a]$ for the preorder $s$ and of the set
$\{b| a<_r b\mbox{ and }a\simeq_s b\}$, and the second of these sets is contained in the first.

\end{sublemma}


\subsection{The ascent partition of a preorder}

\begin{subdefi}[AscentPartition\_aux]
If $r$ is a linear order on $\alpha$ and $s$ is a preorder on $\alpha$, the \emph{ascent partition} of $s$ (with respect to $r$)
is the relation $AP(r,s)$ defined by $a\le_{AP(r,s)}b$ if $a\le_s b$, or $b\le_s a$ and the identity on the interval $[b,a]$ for $s$
is striclty monotone for the preorders $s$ and $r$.
(The last condition means that, if $c,d$ are elements of $\alpha$ such that $a\le_s c <_s d \le_s b$, then $c<_r d$.)

\end{subdefi}

\begin{sublemma}[AscentPartition\_aux\_refl \& AscentPartition\_aux\_trans \& AscentPartition\_aux\_total]
If $s$ is a total preorder, then $AP(r,s)$ is a total preorder.

\end{sublemma}

\begin{subdefi}[AscentPartition]
The relation $AP(r,s)$ as a preorder (for $s$ a total preorder).

\end{subdefi}

\begin{sublemma}[AscentPartition\_is\_total]
If $s$ is a total preorder, then the preorder $AP(r,s)$ is total.

\end{sublemma}

\begin{sublemma}[AscentPartition\_is\_greater]
If $s$ is a total preorder, then $AP(r,s)$ is greater than or equal to $s$.

\end{sublemma}


\subsection{Interactions between these two constructions}

\begin{sublemma}[AscentPartition\_comp]
If $r$ is a linear order and $s$ is a total preorder, then $AP(r,s)=AP(r,LO(r,s))$.

\end{sublemma}

\begin{sublemma}[LinearOrder\_of\_AscentPartition]
If $r$ is a linear order and $s$ is a total preorder, then $LO(r,s)=LO(r,AP(r,s))$.

\end{sublemma}

\begin{sublemma}[LinearOrder\_of\_total\_preorder\_and\_linear\_order\_is\_constant\_on\_interval\_aux \& LinearOrder\_of\_total\_preorder\_and\_linear\_order\_is\_constant\_on\_interval]
Let $r$ be a linear order on $\alpha$ and $s,t,u$ be total preorders on $\alpha$ such that $s\le t\le u$ and $LO(r,s)=LO(r,u)$. Then $LO(r,s)=LO(r,t)$.

\end{sublemma}

\begin{sublemma}[LinearOrder\_of\_total\_preorder\_and\_linear\_order\_on\_ascent\_interval]
Let $r$ be a linear order on $\alpha$ and $s,t$ be total preorders on $\alpha$ such that $s\le t\le AP(r,s)$. Then $LO(r,s)=LO(r,t)$.  

\end{sublemma}


\begin{sublemma}[LinearOrder\_of\_total\_preorder\_and\_linear\_order\_on\_ascent\_interval']
Let $r$ be a linear order on $\alpha$ and $s,t$ be total preorders on $\alpha$ such that $s\le t\le AP(r,s)$ and $s$ is a linear order. Then $s=LO(r,t)$.  

\end{sublemma}

\begin{sublemma}[LinearOrder\_of\_total\_preorder\_and\_linear\_order\_fibers]
Let $r$ be a linear order on $\alpha$ and $s,t$ be preorder on $\alpha$ such that $s$ is a linear order, $t$ is total and $LO(r,t)=s$. Then we have
$s\le t\le AP(r,s)$.

\end{sublemma}

\begin{sublemma}[AscentPartition\_fibers]
Let $r$ be a linear order on $\alpha$ and $s,t$ be preorder on $\alpha$ such that $s$ is a linear order and $t$ is total.
Then $AP(r,s)=AP(r,t)$ if and only if $s\le t\le AP(r,s)$.

\end{sublemma}

\begin{sublemma}[AscentPartition\_fibers']
Let $r$ be a linear order on $\alpha$ and $s,t$ be preorder on $\alpha$ such that $s$ is a linear order and $t$ is total.
Then $AP(r,s)=AP(r,t)$ if and only if $s=LO(r,t)$.

\end{sublemma}


\subsection{Eventually trivial partitions}

This part does not seem to be useful anymore.

\begin{subdefi}[EventuallyTrivialLinearOrderedPartitions]
Let $s$ be a linear order on $\alpha$. A linearly ordered partitions $s$ is called \emph{essentially trivial} if there exists
$a$ in $\alpha$ such that, for all $b,c$ in $\alpha$, if $b,c\ge_r a$, we have $b\le_s c$.

\end{subdefi}

\begin{sublemma}[EventuallyTrivial\_is\_finite]
Let $r$ be a linear order on $\alpha$ that is locally finite with a smallest element, and let $s$ be an essentially trivial linearly
ordered partition. Then the antisymmetrization of $s$ is finite.

\end{sublemma}

\begin{sublemma}[EventuallyTrivial\_IsUpperSet]
Let $r$ be a linear order on $\alpha$. Then eventually trivial linearly ordered partitions form an upper set.

\end{sublemma}

\begin{sublemma}[Finite\_is\_EventuallyTrivial]
If $\alpha$ is finite nonempty, then every linearly ordered partition is eventually trivial (for any choice of linear order $r$).

\end{sublemma}


\subsection{Some calculations}

\begin{sublemma}[AscentPartition\_fixed\_linear\_order]
Let $r$ be a linear order on $\alpha$. Then $AP(r,r)$ is the trivial preorder on $\alpha$.

\end{sublemma}

\begin{sublemma}[Preorder\_lt\_and\_AscentPartition\_ge\_implies\_LinearOrder\_le]
Let $r$ be a linear order and $s$ be a total preorder. For all $a,b$ in $\alpha$ such that
$a<_s b$ and $b\le_{AP(r,s)} a$, we have $a\le_r b$.

\end{sublemma}

\begin{sublemma}[AscentPartition\_trivial\_implies\_fixed\_linear\_order]
Let $r$ be a linear order and $s$ be a preorder. If $s$ is a linear order and $AP(r,s)$ is the trivial preorder, then $s=r$. 

\end{sublemma}

\begin{sublemma}[AscentPartition\_dual\_fixed\_linear\_order]
Let $r$ be a linear order on $\alpha$. Then $AP(r, dual(r))$ is equal to $dual(r)$.

\end{sublemma}

We want to prove the converse of the last lemma under some conditions on $s$. This requires some preparation.

\begin{subdefi}[ReverseProductOrder]
If $s$ is a preorder on $\alpha$, then this is the preorder on $\alpha\times\alpha$ that is the product of the dual of $\alpha$ and of $\alpha$.

\end{subdefi}

\begin{sublemma}[ReverseProductOrder\_lt1 \& ReverseProductOrder\_lt2]
Let $s$ be a preorder on $\alpha$ and $a,b,c$ be elements of $\alpha$. If $a<_s b$, then $(b,c)<(a,c)$ for the reverse product order, and if
$b<_s c$, then $(a,b)<(a,c)$ for the reverse product order.

\end{sublemma}

\begin{sublemma}[Exists\_smaller\_noninversion]
Let $r$ be a linear order and $s$ be a preorder. We suppose that $s$ is a linear order and that $AP(r,s)=s$. If $a,b$ are elements of $\alpha$
such that $a<_r b$ and $a<_s b$, then there exist $c,d$ in $\alpha$ such that $c<_r d$, $c<_s d$ and $(c,d)$ is strictly smaller than $(a,b)$ for
the reverse product preorder defined by $s$.

\end{sublemma}

\begin{sublemma}[AscentPartition\_linear\_implies\_dual\_linear\_order]
Let $r$ be a linear order and $s$ be a preorder. We suppose that $s$ is a linear order, that $AP(r,s)=s$ and that $s$ is locally finite.
Then is equal to the dual of $r$.

\end{sublemma}
\section{The weak Bruhat order on linear orders}

We fix a set $\alpha$.

\subsection{Inversions}

\begin{subdefi}[Inversions]
Let $r$ and $s$ be relations on $\alpha$. The set $Inv(r,s)$ of inversions of $s$ relative to $r$ is the set of pair $(a,b)$ of
elements of $\alpha$ such that $r(a,b)$ and $s(b,a)$ hold. 

\end{subdefi}

If $r$ (resp. $s$) is a preorder, we write $Inv(r,s)$ for $Inv(<_r,s)$ (resp. $Inv(r,<_s))$).

\begin{sublemma}[Inversions\_antitone]
Let $r$ be a relation on $\alpha$ and $s,t$ be preorders on $\alpha$ such that $s\le t$ and $s$ is total. Then
$Inv(r,s)\subset Inv(r,t)$.

\end{sublemma}

\begin{sublemma}[LinearOrders\_eq\_iff\_no\_inversions]
Let $r,r'$ be linear orders on $\alpha$. Then $r=r'$ if and only $Inv(r,r')$ is empty.

\end{sublemma}

\begin{sublemma}[Inversions\_of\_associated\_linear\_order]
Let $r$ be a linear order on $\alpha$ and $s$ be a total preorder on $\alpha$. Then $Inv(r,s)=Inv(r,LO(r,s))$.

\end{sublemma}

\begin{sublemma}[Inversions\_of\_AscentPartition]
Let $r$ be a linear order on $\alpha$ and $s$ be a total preorder on $\alpha$. Then $Inv(r,s)=Inv(r,AP(r,s))$.

\end{sublemma}

\begin{sublemma}[Inversions\_dual\_order]
Let $r$ be a linear order on $\alpha$ and $a,b$ be elements of $\alpha$. Then $(a,b)\in Inv(r,dual(r))$ if and only
if $a<_r b$.

\end{sublemma}

\begin{sublemma}[Inversions\_determine\_linear\_order\_aux \& Inversions\_determine\_linear\_order]
Let $r,s_1,s_2$ be linear orders on $\alpha$. If $Inv(r,s_1)=Inv(r,s_2)$, then $s_1=s_2$.

\end{sublemma}


\subsection{Weak Bruhat order}

\begin{subdefi}[WeakBruhatOrder]
Let $r$ be a linear order on $\alpha$. We define the weak Bruhat order (relative to $r$) on linear orders by
setting $s<_{wB}t$ if $Inv(r,s)\subset Inv(r,t)$.

\end{subdefi}

\begin{sublemma}[WeakBruhatOrder\_iff]
Let $r,s_1,s_2$ be linear orders on $\alpha$. Then $s_1\le s_2$ for the weak Bruhat order realtive to $r$ if and only
$Inv(s_1,s_2)=Inv(r,s_2) \setminus Inv(r,s_1)$.

\end{sublemma}

\begin{sublemma}[WeakBruhatOrder\_iff']
Let $r,s_1,s_2$ be linear orders on $\alpha$. Then $s_1\le s_2$ for the weak Bruhat order realtive to $r$ if and only
$Inv(r,s_2)=Inv(r,s_1) \cup Inv(s_1,s_2)$.

\end{sublemma}

\begin{sublemma}[WeakBruhatOrder\_smallest]
Let $r,s$ be linear orders on $\alpha$. Then $r\le s$ for the weak Bruhat order relative to $r$.

\end{sublemma}

\begin{sublemma}[WeakBruhatOrder\_greatest]
Let $r,s$ be linear orders on $\alpha$. Then $s\le dual(r)$ for the weak Bruhat order relative to $r$.

\end{sublemma}


\subsection{Finite chains for the weak Bruhat order}

\begin{sublemma}[Finite\_inversions\_finite\_inversion\_interval]
If $s,t$ are linear orders on $\alpha$ such that $Inv(s,t)$ is finite and $a,b$ are elements of $\alpha$ such that
$(a,b)\in Inv(s,t)$, then the closed interval $[a.b]$ for $s$ is finite.

\end{sublemma}

\begin{sublemma}[Finite\_inversions\_exists\_elementary\_inversion\_rec \& Finite\_inversions\_exists\_elementary\_inversion]
If $s,t$ are linear orders on $\alpha$ such that $Inv(s,t)$ is finite and nonempty, then there exist $a,b$ in $\alpha$ such that
$(a,b)\in Inv(s,t)$ and $b$ covers $a$ for $s$.

\end{sublemma}

\begin{subdefi}[Transposition]
If $a,b$ are elements of $\alpha$, we define the transposition $\tau_{a,b}$: it is the map from $\alpha$ to $\alpha$ that exchanges
$a$ and $b$ and leaves all other elements fixed.

\end{subdefi}

\begin{sublemma}[Transposition\_is\_involutive]
Let $a,b,x$ be elements of $\alpha$. Then $\tau_{a,b}(\tau_{a,b}(x))=x$.

\end{sublemma}

\begin{sublemma}[Transposition\_is\_injective]
Let $a,b$ be elements of $\alpha$. Then $\tau_{a,b}$ is injective.

\end{sublemma}

\begin{subdefi}[TransposedPreorder]
If $a,b$ are elements of $\alpha$ and $s$ is a preorder on $\alpha$, then the transposed preorder $\tau_{a,b}(s)$ is the lift of
$s$ via $\tau_{a,b}$.

\end{subdefi}

\begin{subdefi}[Transposed\_of\_linear\_is\_linear]
If $a,b$ are elements of $\alpha$ and $s$ is a preorder on $\alpha$ that is a linear order, then $\tau_{a,b}(s)$ is a linear order.

\end{subdefi}

\begin{subdefi}[CoveringElementBruhatOrder]
Let $s,t$ be linear orders on $\alpha$ such that $Inv(s,t)$ is finite and nonempty. We define a linear order $cov(s,t)$ on $\alpha$ by taking the
transposed preorder of $s$ by an arbitrary $(a,b)\in Inv(s,t)$ such that $b$ covers $a$ for $s$.

\end{subdefi}

\begin{sublemma}[CoveringElementBruhatOrder\_Inversions1]
Let $s,t$ be linear orders on $\alpha$ such that $Inv(s,t)$ is finite and nonempty. Then
$Inv(s,cov(s,t))$ is equal to $\{(a,b)\}$, where $(a,b)$ is the element of $Inv(s,t)$ that was used to define $cov(s,t)$. 

\end{sublemma}

\begin{sublemma}[CoveringElementBruhatOrder\_Inversions2]
Let $s,t$ be linear orders on $\alpha$ such that $Inv(s,t)$ is finite and nonempty. Then
$Inv(cov(s,t),t)$ is equal to $Inv(s,t)\setminus\{(a,b)\}$, where $(a,b)$ is the element of $Inv(s,t)$ that was used to define $cov(s,t)$. 

\end{sublemma}

\begin{sublemma}[CoveringElementBruhatOrder\_Inversions3]
Let $s,t$ be linear orders on $\alpha$ such that $Inv(s,t)$ is finite and nonempty. Then
$Inv(cov(s,t),t)\subset Inv(s,t)$.

\end{sublemma}

\begin{sublemma}[CoveringElementBruhatOrder\_Inversions4]
Let $r,s,t$ be linear orders on $\alpha$ such that $Inv(s,t)$ is finite and nonempty and $s\le t$ for the weak Bruhat relative to $r$.
Then $Inv(r,cov(s,t))$ is equal $Inv(r,s)\cup\{(a,b)\}$, where $(a,b)$ is the element of $Inv(s,t)$ that was used to define $cov(s,t)$. 

\end{sublemma}

\begin{sublemma}[CoveringElementBruhatOrder\_covering]
Let $r,s,t$ be linear orders on $\alpha$ such that $Inv(s,t)$ is finite and nonempty and $s\le t$ for the weak Bruhat relative to $r$.
Then $cov(s,t)$ covers $s$ for the weak Bruhat order relative to $r$

\end{sublemma}

\begin{sublemma}[CoveringElementBruhatOrder\_smaller]
Let $r,s,t$ be linear orders on $\alpha$ such that $Inv(s,t)$ is finite and nonempty and $s\le t$ for the weak Bruhat relative to $r$.
Then $cov(s,t)\le t$ for the weak Bruhat order relative to $r$

\end{sublemma}




\section{Finite linearly ordered partitions}

We fix a set $\alpha$.

\subsection{Almost finite linearly ordered partitions}

\begin{subdefi}[AFLOPartitions]
An \emph{almost finite linearly ordered partition} (AFLO partition) of $\alpha$ is a total preorder $s$ on $\alpha$ such that:
\begin{itemize}
\item[(1)] For every element $a$ of $\alpha$, the half-infinite interval $]\leftarrow,a]$ for $s$ is equal to $\alpha$
or finite.

\item[(2)] The antisymmetrization of $s$ is finite.

\end{itemize}
\end{subdefi}

\begin{subdefi}[AFLOPartitions.PartialOrder]
We define a partial order on AFLO partitions by restricting the partial order on preorders.

\end{subdefi}

\begin{sublemma}[AFLO\_has\_finite\_blocks]
If $s$ is an AFLO partition and $x$ is any nonmaximal element of the antisymmetrization of $s$, then the preimage of
$x$ in $\alpha$ is finite.

\end{sublemma}

\begin{subdefi}[AFLOPartition\_is\_ELF]
We define a structure of essentially locally finite preorder on any AFLO partition.

\end{subdefi}


\subsection{Sets of lower sets of AFLO partitions}

We want to characterize the image of the set of AFLO partitions by the map $preorderToPowerset$.

\begin{subdefi}[AFLOPowerset]
We define AFLOPowerset to be the set of finite subsets $E$ of the powerset of $\alpha$ that are totally ordered by inclusion and such
that, for every $X\in E$, we have $X$ nonempty, not equal to $\alpha$ and finite.

\end{subdefi}

\begin{sublemma}[AFLO\_preorderToPowerset\_finite]
If $s$ is an AFLO partition, then $preorderToPowerset(s)$ is a finite set.

\end{sublemma}

\begin{subdefi}[preorderToPowersetFinset]
For $s$ an AFLO partitions, $preorderToPowersetFinset(s)$ is $preorderToPowerset(s)$ seen as a finite subset of $\alpha$.

\end{subdefi}

\begin{subdefi}[AFLO\_preorderToPowerset]
For every AFLO partition $s$, the finite set $preorderToPowersetFinset(s)$ is in AFLOPowerset.

\end{subdefi}

\begin{sublemma}[AFLO\_powersetToPreorder]
If $E$ is an element of AFLOPowerset, then $powersetToPreorder(E)$ is an AFLO partition.

\end{sublemma}
\section{Decomposable abstract simplicial complexes}

We fix a set $\alpha$ and an abstract simplicial complex $K$ on $\alpha$.

\subsection{Definition of decomposability}

\begin{subdefi}[IsDecomposition]
Let $R$ be a function from the set of facets of $K$ to the set of finite subset of $\alpha$ and $DF$ be a function
from the set of faces of $K$ to the set of facets of $K$.
We say that $R$ and $DF$ define a decomposition of $K$ if:
\begin{itemize}
\item[(1)] For every facet $s$ of $K$, we have $R(s)\subset s$.
\item[(2)] For every face $s$ of $K$ and every facet $t$ of $K$, we have $R(t)\subset s\subset t$ if and only $DF(s)=t$.

\end{itemize}
\end{subdefi}

In what follows, we just say "$(R,DF)$ is a decomposition of $K$".

\begin{sublemma}[Decomposition\_DF\_bigger\_than\_source]
Let $(R,DF)$ be a decomposition of $K$.
Then, for every face $s$ of $K$, we have $s\subset DF(s)$.

\end{sublemma}

\begin{sublemma}[Decomposition\_is\_union]
Let $(R,DF)$ be a decomposition of $K$.
Then, for every face $s$ of $K$, there exists a facet $t$ of $K$ such that $R(t)\le s\le t$.

\end{sublemma}

\begin{sublemma}[Decomposition\_is\_disjoint]
Let $(R,DF)$ be a decomposition of $K$ and $s$ be a face of $K$.
If $t_1,t_2$ are facets of $K$ such that $R(t_1)\le s\le t_1$ and $R(t_2)\le s\le t_2$, then
$t_1=t_2$.

\end{sublemma}

\begin{sublemma}[Decomposition\_DF\_of\_a\_facet]
Let $(R,DF)$ be a decomposition of $K$ and $s$ be a facet of $K$. Then $DF(s)=s$.

\end{sublemma}

\begin{sublemma}[Decomposition\_image\_of\_R]
Let $(R,DF)$ be a decomposition of $K$ and $s$ be a facet of $K$. If $R(s)\ne\varnothing$, then
$R(s)$ is a face of $K$.

\end{sublemma}

\begin{sublemma}[Decomposition\_image\_of\_R']
Let $(R,DF)$ be a decomposition of $K$ and $s$ be a facet of $K$. Then $R(s)=\varnothing$ or
$R(s)$ is a face of $K$.

\end{sublemma}

\begin{sublemma}[Decomposition\_SF\_composed\_with\_R]
Let $(R,DF)$ be a decomposition of $K$ and $s$ be a facet of $K$. If $R(s)\ne\varnothing$, then $s=DF(R(s))$.

\end{sublemma}

\begin{sublemma}[Decomposition\_R\_determines\_DF]
Let $(R,DF_1)$ and $(R,DF_2)$ be decompositions of $K$. Then $DF_1=DF_2$.

\end{sublemma}


\subsection{Intervals of a decomposition}

\begin{subdefi}[Interval]
If $s$ is empty of a face of $K$ and $t$ is a face of $K$, we define $Interval(s,t)$ as the finite set of faces $u$ of $K$
such that $s\subset u\subset t$.

\end{subdefi}

\begin{subdefi}[DecompositionInterval]
Let $(R,DF)$ be a decomposition of $K$ and $s$ be a facet of $K$. The corresponding \emph{decomposition interval}
is $Interval(R(s),s)$.

\end{subdefi}

\begin{subdefi}[DecompositionInterval\_def]
Let $(R,DF)$ be a decomposition of $K$ and $s$ be a facet of $K$. If $t$ is a face of $K$, then $t$ is in the decomposition
interval of $s$ if and only $R(s)\subset t\subset t$.

\end{subdefi}

\begin{subdefi}[DecompositionInterval\_eq]
Let $(R,DF)$ be a decomposition of $K$ and $s$ be a facet of $K$. If $t$ is a face of $K$, then $t$ is in the decomposition
interval of $s$ if and only $DF(t)=s$.

\end{subdefi}

\begin{sublemma}[Decomposition\_DF\_determines\_R\_intervals]
Let $(R_1,DF)$ and $(R_2,DF)$ be decompositions of $K$. Then, for every facet $s$ of $K$, we have
$Interval(R_1(s),s)=Interval(R_2(s),s)$.

\end{sublemma}


\subsection{Compatible partial orders on facets}

\begin{subdefi}[CompatibleOrder]
Let $DF$ be a map from the set of faces of $K$ to the set of facets of $K$ and $r$ be a partial order on the set of
facets of $K$. We say that $r$ is \emph{compatible} with $DF$ if, for every face $s$ of $K$ and every facet $t$ of $K$,
if $s\le t$, then $DF(s)\le_r t$.

\end{subdefi}

\begin{sublemma}[OldFacesCompatibleOrder]
Let $DF$ be a map from the set of faces of $K$ to the set of facets of $K$ and $r$ be a partial order on the set of
facets of $K$ that is compatible with $DF$. Let $s$ be a facet of $K$ and $t$ be a face of $K$ such that $t\le s$ and
$t\le DF(t)$. Then $t$ is not in the complex of old faces of $s$ (relative to $r$) if and only if $DF(t)=s$.

\end{sublemma}

\begin{sublemma}[OldFacesDecomposition]
Let $(R,DF)$ be decomposition of $K$ and $r$ be a partial order on the set of
facets of $K$ that is compatible with $DF$. Let $s$ be a facet of $K$ and $t$ be a face of $K$ such that $t\le s$.
Then $t$ is not in the complex of old faces of $s$ (relative to $r$) if and only if $t$ is in the decomposition interval
corresponding to $s$.

\end{sublemma}

\begin{sublemma}[OldFacesDecomposition']
Let $(R,DF)$ be decomposition of $K$ and $r$ be a partial order on the set of
facets of $K$ that is compatible with $DF$. Let $s$ be a facet of $K$ and $t$ be a face of $K$ such that $t\le s$.
Then $t$ is not in the complex of old faces of $s$ (relative to $r$) if and only if $R(s)\subset t$.

\end{sublemma}

\begin{sublemma}[OldFacesDecomposition\_faces]
Let $(R,DF)$ be decomposition of $K$ and $r$ be a partial order on the set of
facets of $K$ that is compatible with $DF$. Let $s$ be a facet of $K$ and $t$ be a face of $K$ such that $t\le s$.
Then $t$ is in the complex of old faces of $s$ (relative to $r$) if and only if $R(s)$ is not included in $t$.

\end{sublemma}

\begin{sublemma}[OldFacesDecomposition\_empty\_iff]
Let $(R,DF)$ be decomposition of $K$ and $r$ be a partial order on the set of
facets of $K$ that is compatible with $DF$. Let $s$ be a facet of $K$. Then the complex of old faces of $s$ relative
to $r$ is empty if and only the decomposition interval of $s$ is equal to the half-infinite interval $]\leftarrow,s]$.

\end{sublemma}

\begin{sublemma}[OldFacesDecompositionDimensionFacets]
Let $(R,DF)$ be decomposition of $K$ and $r$ be a partial order on the set of
facets of $K$ that is compatible with $DF$. Let $s$ be a facet of $K$ and $t$ a facet of the complex of old faces of $s$ relative
to $r$. Then the cardinality of $t$ is equality to the cardinality of $s$ minus $1$.

\end{sublemma}

\begin{sublemma}[OldFacesDecompositionIsPure]
Let $(R,DF)$ be decomposition of $K$ and $r$ be a partial order on the set of
facets of $K$ that is compatible with $DF$. Let $s$ be a facet of $K$. Then the complex of old faces of $s$ relative
to $r$ is pure.

\end{sublemma}

\begin{sublemma}[OldFacesDecompositionDimension]
Let $(R,DF)$ be decomposition of $K$ and $r$ be a partial order on the set of
facets of $K$ that is compatible with $DF$. Let $s$ be a facet of $K$. Then the complex of old faces of $s$ relative
to $r$ is of dimension $card(s)-2$.

\end{sublemma}


\subsection{$\pi_0$ and homology facets}

\begin{subdefi}[IsPi0Facet]
Let $(R,DF)$ be decomposition of $K$ and $s$ be a facet of $K$. We say that $s$ is a \emph{$\pi_0$ facet} if the decomposition
interval of $s$ is equal to the half-infinite interval $]\leftarrow,s]$.

\end{subdefi}

\begin{subdefi}[IsHomologyFacet]
Let $(R,DF)$ be decomposition of $K$ and $s$ be a facet of $K$. We say that $s$ is a \emph{homology facet} if it is not a $\pi_0$ facet
and if the decomposition interval of $s$ is equal to the singleton $\{s\}$.

\end{subdefi}

\begin{sublemma}[Vertex\_IsPi0Facet]
Let $(R,DF)$ be decomposition of $K$ and $s$ be a facet of $K$. If the cardinality of $s$ is equal to $1$, then $s$ is a $\pi_0$ facet.

\end{sublemma}

\begin{sublemma}[IsPi0Facet\_iff]
Let $(R,DF)$ be decomposition of $K$ and $s$ be a facet of $K$. Then $s$ is a $\pi_0$ facet if and only if $R(s)$ is empty or the
cardinality of $s$ is equal to $1$.

\end{sublemma}

\begin{sublemma}[IsHomologyFacet\_iff]
Let $(R,DF)$ be decomposition of $K$ and $s$ be a facet of $K$. Then $s$ is a homology facet if and only if $R(s)=s$ is empty and the
cardinality of $s$ is $>1$.

\end{sublemma}











\section{Shellable abstract simplicial complexes}

We fix a set $\alpha$ and an abstract simplicial complex $K$ on $\alpha$.

\subsection{Shelling orders}

\begin{subdefi}[IsShellingOrder]
A linear order $r$ on the facets of $K$ is called a \emph{shelling order} if it is well-founded and if, for
every facet $s$ of $K$, the complex of old faces corresponding to $s$ is either empty or pure of dimension
$card(s)-2$.

\end{subdefi}


\subsection{The restriction map}

\begin{subdefi}[ShellingOrderRestriction\_aux \& ShellingOrderRestriction]
The "restriction map" for a partial order $s$ on the facets of $K$. It sends a facet $s$ of $K$ to the set
of vertices $a$ of $K$ such that $a\in s$ and $s\setminus\{a\}$ is a face of the complex of old faces.

\end{subdefi}

\begin{sublemma}[ShellingOrderRestriction\_mem]
Let $r$ be a partial order on the facets of $K$, $s$ be a facet of $K$ and $a$ be an element of $\alpha$. Then $a$ is
in the image of $s$ by the restriction map if and only if $a\in s$, $s\ne\{a\}$ and there exists a facet $u$ of $K$ such
that $u<_r s$ and $s\setminus\{a\}\subset u$.

\end{sublemma}

\begin{sublemma}[ShellingOrderRestriction\_smaller]
Let $r$ be a partial order on the facets of $K$ and $s$ be a facet of $K$. Then the image of $s$ by the restriction map is
contained in $s$.

\end{sublemma}

\begin{sublemma}[not\_containing\_restriction\_is\_old\_face]
Let $r$ be a partial order on the facets of $K$, $s$ be a facet of $K$ and $t$ be a face of $K$ such that $t\subset s$ and
the image of $s$ by the restriction map is not contained in $t$. Then $t$ is a face of the complex of old faces defined by $s$.

\end{sublemma}

\begin{sublemma}[old\_face\_does\_not\_contain\_restriction]
Let $r$ be a partial order on the facets of $K$, $s$ be a facet of $K$ and $t$ be a finite subset of $\alpha$. We suppose that
the complex of old faces defined by $s$ is pure of dimension $card(s)-2$ and that $t$ is a face of this complex. Then
the image of $s$ by the restriction map is not contained in $t$.

\end{sublemma}


\subsection{The smallest facet map}

\begin{subdefi}[ExistsFacet]
We say that $K$ satisfies condition $ExistsFacet$ if every face of $K$ is contained in a facet.

\end{subdefi}

\begin{sublemma}[Noetherian\_ExistsFacet]
If the face poset of $K$ is Noetherian, then $K$ satisfies condition $ExistsFacet$.

\end{sublemma}

\begin{subdefi}[ShellingOrderSmallestFacet]
Let $r$ be a well-order on the facets of $K$; we suppose that $K$ satisfies condition $ExistsFacet$.
The smallest facet map sends a face $s$ of $K$ to the smallest (for $r$) facet containing $s$.

\end{subdefi}

\begin{sublemma}[ShellingOrderSmallestFacet\_bigger]
Let $r$ be a well-order on the facets of $K$; we suppose that $K$ satisfies condition $ExistsFacet$.
Then every face of $K$ is contained in its image by the smallest facet map.

\end{sublemma}

\begin{sublemma}[ShellingOrderSmallestFacet\_smallest]
Let $r$ be a well-order on the facets of $K$; we suppose that $K$ satisfies condition $ExistsFacet$.
Let $s$ be a face of $K$ and $u$ be a facet of $K$ such that $s\le u$. Then $u$ is bigger for $r$ or equal
to the image of $s$ by the smallest facet map.

\end{sublemma}


\subsection{Shellability vs decomposability}

\begin{sublemma}[ShellableIsDecomposable]
Let $r$ be a shelling order on the facets of $K$, and suppose that $K$ satisfies condition $ExistsFacet$. Then the restriction
map and the smallest facet map form a decomposition of $K$.

\end{sublemma}

\begin{sublemma}[ShellableofDecomposable]
Let $(R,DF)$ be a decomposition of $K$, and $r$ be a well-order on the facets of $K$ that is compatible with $DF$. Then $r$ is a
shelling order.

\end{sublemma}

\begin{sublemma}[ExistsFacetofDecomposable]
Let $(R,DF)$ be a decomposition of $K$. Then $K$ satifies condition $ExistsFacet$.

\end{sublemma}

\begin{sublemma}[ShellableofDecomposable\_smallestfacet]
Let $(R,DF)$ be a decomposition of $K$, and $r$ be a well-order on the facets of $K$ that is compatible with $DF$. Then the smallest facet
map for $r$ is equal to $DF$.

\end{sublemma}

\begin{sublemma}[ShellableofDecomposable\_intervals]
Let $(R,DF)$ be a decomposition of $K$, and $r$ be a well-order on the facets of $K$ that is compatible with $DF$. For every facet
$s$ of $K$. the decomposition interval of $s$ for $(R,DF)$ is equal to its decomposition interval for the decomposition defined by
the shelling order $r$.

\end{sublemma}
\section{Euler-Poincaré characteristic of a finite simplicial complex}

We fix a set $\alpha$ and an abstract simplicial complex $K$ on $\alpha$.

\subsection{Definition}

\begin{subdefi}[FacesFinset]
If $K$ is a finite complex, we define the finite set of faces of $K$.

\end{subdefi}

\begin{subdefi}[FacetsFinset]
If $K$ is a finite complex, we define the finite set of facets of $K$.

\end{subdefi}

\begin{subdefi}[EulerPoincareCharacteristic]
If $K$ is a finite complex, its Euler-Poincaré characteristic is the sum over all faces $s$ of $K$ of
$(-1)^{card(s)-1}$.

\end{subdefi}

\begin{sublemma}[EulerPoincareCharacteristic\_ext]
If $K$ and $L$ are finite abstract simplicial complexes with equal sets of faces, then their Euler-Poincaré
characteristic are equal.

\end{sublemma}

\subsection{The case of decomposable complexes}

\begin{subdefi}[$\pi_0$Facets]
Let $(R,DF)$ be a decomposition of $K$. We define the set of $\pi_0$ facets of $K$, as a set of finite subsets of $\alpha$.

\end{subdefi}

\begin{subdefi}[HomologyFacets]
Let $(R,DF)$ be a decomposition of $K$. We define the set of homology facets of $K$, as a set of finite subsets of $\alpha$.

\end{subdefi}

\begin{sublemma}[$\pi_0$Facets\_finite]
Let $(R,DF)$ be a decomposition of $K$, and suppose that $K$ is finite. Then the set of $\pi_0$ facets of $K$ is finite.

\end{sublemma}

\begin{sublemma}[HomologyFacets\_finite]
Let $(R,DF)$ be a decomposition of $K$, and suppose that $K$ is finite. Then the set of homology facets of $K$ is finite.

\end{sublemma}

We now introduce some auxiliary definitions that we will need in the calculation.

\begin{subdefi}[DFe]
If $DF$ is a map from the set of faces of $K$ to the set of facets of $K$, we extend $DF$ to a map $DFe$ from the set of finite
subsets of $\alpha$ to itself, by sending a finite subset $s$ to $DF(s)$ if $s$ is a face of $K$, and to $\varnothing$ otherwise.

\end{subdefi}

\begin{subdefi}[Quotient\_DFe\_to\_finset]
Let $DF$ be a map from the set of faces of $K$ to the set of facets of $K$. We define a map from the quotient of $\alpha$ by
the equivalence relation $\Ker(DFe)$ to the set of finite subsets of $\alpha$ by sending the equivalence class of a finite
set $s$ to $DFe(s)$.

\end{subdefi}

\begin{sublemma}[Quotient\_DFe\_to\_finset\_is\_facet\_aux \& Quotient\_DFe\_to\_finset\_is\_facet]
Let $DF$ be a map from the set of faces of $K$ to the set of facets of $K$. Then the map from $\alpha/\Ker(DFe)$ to the set of finite
subsets of $\alpha$ sends the class of a face of $K$ to a facet of $K$.

\end{sublemma}

\begin{subdefi}[DecompositionInterval']
If $R$ is a map from the set of facets of $K$ to the set of finite subsets of $\alpha$ and $s$ a facet of $K$, we define the
interval $[R(s),s]$ as a finite set of nonempty finite subsets of $\alpha$.

\end{subdefi}

\begin{sublemma}[ComparisonIntervals]
Let $(R,DF)$ be a decomposition of $K$, $s$ be a facet of $K$ and $t$ be a finite subset of $\alpha$.
Then $t$ is in the decomposition interval of the previous definition if and only if it is a face of $K$ and a member of the 
decomposition interval defined by $s$.

\end{sublemma}

\begin{subdefi}[Sum\_on\_DecompositionInterval]
Let $R$ be a map from the set of facets of $K$ to the set of finite subsets of $\alpha$ and $s$ be a finite subset of $\alpha$.
We define a "sum on the decomposition interval corresponding to $s$" in the following way: if $s$ is a facet of $K$, then it is the
sum on the elements $t$ of the decomposition interval of $(-1)^{card(t)-1}$, otherwise it is $0$.

\end{subdefi}

\begin{sublemma}[ComparisonFunctionsonQuotient]
Ler $(R,DF)$ be a decomposition of $K$ and $x$ be an element of $\alpha/\Ker(DFe)$. Suppose that $K$ is a finite complex and
that $x$ is the class of a face of $K$. Then the sum over $t$ in the class $x$ of $(-1)^{card(t)-1}$ is equal to the
"sum on the decomposition interval" function applied to the image of $x$ by the quotient of the map $DFe$.

\end{sublemma}

\begin{sublemma}[Quotient\_DFe\_to\_finset\_inj]
Let $DF$ be a map from the set of faces of $K$ to the set of facets of $K$, and suppose that $K$ is a finite complex.
Then the map from $\alpha/\Ker(DFe)$ to the set of finite subsets of $\alpha$ defined by $DFe$ is injective.
\footnote{Why do we need $K$ to be finite ? Surely this is a very general fact. Same remark about the next lemma.
(I know why the condition is there in the Lean file, it's because I use the "finset" versions of the sets of faces and facets, but this
is not a good reason.)}
\end{sublemma}

\begin{sublemma}[Quotient\_DFe\_to\_finset\_surj]
Let $(R,DF)$ be a decomposition of $K$, and suppose that $K$ is a finite complex.
Then every facet of $K$ is in the image of the map from $\alpha/\Ker(DFe)$ to the set of finite subsets of $\alpha$ defined by $DFe$.
\end{sublemma}

\begin{subdefi}[BoringFacets]
Let $(R,DF)$ be a decomposition of $K$. The set of boring facets is the set of finite subsets of $\alpha$ that are facets of $K$ but
are neither $\pi_0$ facets nor homology facets.

\end{subdefi}

\begin{sublemma}[BoringFacets\_finite]
Let $(R,DF)$ be a decomposition of $K$. If $K$ is finite, then the set of boring facets of $K$ is finite.

\end{sublemma}

\begin{sublemma}[every\_facet\_is\_boring\_or\_interesting \& boring\_is\_not\_interesting]
Let $(R,DF)$ be a decomposition of $K$, and suppose that $K$ is finite. Then the set of facets of $K$, seen as a finite set of
finite subsets of $K$, is the disjoint union of the set of boring facets and the set of facets that are $\pi_0$ or homology facets.

\end{sublemma}

\begin{sublemma}[pi0\_and\_homology\_are\_disjoint]
Let $(R,DF)$ be a decomposition of $K$, and suppose that $K$ is finite. Then the sets of $\pi_0$ and homology facets of $K$, seen
as finite sets of finite subsets of $\alpha$, are disjoint.

\end{sublemma}


\begin{sublemma}[AlternatingSumPowerset]
Let $s$ be a nonempty finite subset of $\alpha$. Then 
the sum of the function $t\mapsto (-1)^{card(t)}$ on the powerset of $s$ is equal to $0$.

\end{sublemma}

\begin{sublemma}[Sum\_on\_FinsetInterval1]
Let $s,t$ be finite subsets of $\alpha$ such that $s\subsetneq t$. Then 
the sum of the function $x\mapsto (-1)^{card(x)}$ on the interval $[s,t]$ is equal to $0$.

\end{sublemma}

\begin{sublemma}[Sum\_on\_FinsetInterval1]
Let $s$ be a nonempty finite subset of $\alpha$. Then 
the sum of the function $x\mapsto (-1)^{card(x)-1}$ on the interval $]\varnothing,s]$ is equal to $1$.

\end{sublemma}

\begin{sublemma}[BoringFacet\_image\_by\_R]
Let $(R,DF)$ be a decomposition of $K$ and $s$ be a boring facet of $K$. Then $R(s)$ is not empty and not equal to $s$.

\end{sublemma}

\begin{sublemma}[Sum\_on\_DecompositionInterval\_BoringFacet]
Let $(R,DF)$ be a decomposition of $K$ and $s$ be a boring facet of $K$. Then the image of $s$ by the sum of the decomposition interval function 
is equal to $0$.

\end{sublemma}

\begin{sublemma}[$\pi_0$Facet\_interval]
Let $(R,DF)$ be a decomposition of $K$ and $s$ be a $\pi_0$ facet of $K$. Then the decomposition interval defined by $s$ is equal to
$]\varnothing,s]$.

\end{sublemma}

\begin{sublemma}[Sum\_on\_DecompositionInterval\_$\pi_0$Facet]
Let $(R,DF)$ be a decomposition of $K$ and $s$ be a $\pi_0$ facet of $K$. Then the image of $s$ by the sum of the decomposition interval function 
is equal to $1$.

\end{sublemma}


\begin{sublemma}[HomologyFacet\_interval]
Let $(R,DF)$ be a decomposition of $K$ and $s$ be a homology facet of $K$. Then the decomposition interval defined by $s$ is equal to
$\{s\}$.

\end{sublemma}

\begin{sublemma}[Sum\_on\_DecompositionInterval\_HomologyFacet]
Let $(R,DF)$ be a decomposition of $K$ and $s$ be a homology facet of $K$. Then the image of $s$ by the sum of the decomposition interval function 
is equal to $(-1)^{card(s)-1}$.

\end{sublemma}

\begin{sublemma}[EulerPoincareCharacteristicDecomposable]
Let $(R,DF)$ be a decomposition of $K$, and suppose that $K$ is finite. Then the Euler-Poincaré characteristic of $K$ is equal to the
cardinality of the set of $\pi_0$ facets plus the sum over all homology facets of the function $s\mapsto (-1)^{card(s)-1}$.

\end{sublemma}


\section{Coxeter complex of a finite symmetric group}

We fix a set $\alpha$. The goal of this section is to define the Coxeter complex (for $\alpha$ arbitrary) and to prove that it is shellable.
This relies on the fact that the Coxeter complex is decomposable, so we will define restriction and distinguished facet maps that depend on
an auxiliary linear order on $\alpha$; the restriction map makes sense in general, but for the distinguished facet map we need to assume that
$\alpha$ is finite (since otherwise the Coxeter complex has no facets).

\subsection{Definition of the Coxeter complex}

For now, we don't assume that $\alpha$ is finite, so we work with almost finite linearly ordered (AFLO) partitions and
their sets of lower sets, AFLOPowerset.

\begin{sublemma}[AFLOPowerset\_down\_closed]
If $E$ is an element of $AFLOPowerset$, then so is every subset of $E$.

\end{sublemma}

\begin{subdefi}[CoxeterComplex]
The \emph{Coxeter complex} is the abstract simplicial complex on the powerset of $\alpha$ whose faces are the nonempty elements of $AFLOPowerset$.

\end{subdefi}

\begin{sublemma}[FacesCoxeterComplex]
Let $s$ be a finite set of subset of $\alpha$. Then $s$ is a face of the Coxeter complex if and only if $s$ is in AFLOPowerset and $s\ne\varnothing$.

\end{sublemma}

\begin{subdefi}[CoxeterComplextoPartitions]
An isomorphism of ordered sets between $AFLOPowerset$ and the dual of the set of AFLO partitions, given the functions $powersetToPreorder$ and $preorderToPowerset$.


\end{subdefi}

\begin{sublemma}[Faces\_powersetToPreordertoPowerset]
If $s$ is a face of the Coxeter complex, then $s=preorderToPowerset(powersetToPreorder(s))$.

\end{sublemma}

\begin{sublemma}[CoxeterComplex\_dimension\_face]
Let $s$ be an element of $AFLOPowerset$, and suppose that $\alpha$ is nonempty. Then the cardinality of $s$ is equal to the
cardinality of the antisymmetrization of $s$ minus $1$.

\end{sublemma}

\begin{sublemma}[twoStepPreorder\_AFLO]
For every element $a$ of $\alpha$, the two-step preorder defined by $a$ is an AFLO partition.

\end{sublemma}

\begin{sublemma}[twoStepPreorder\_in\_CoxeterComplex]
Let $a,b$ be elements of $\alpha$ such that $a\ne b$. Then the image by $preorderToPowerset$ of the two-step preorder defined by $a$
is a face of the Coxeter complex.

\end{sublemma}

\begin{sublemma}[AFLOPartitions\_IsUpperSet]
AFLO partitions form an upper set of the set of preorders of $\alpha$.

\end{sublemma}


\subsection{The restriction map}

\begin{sublemma}[AscentPartition\_respects\_AFLO]
Let $r$ be a linear order on $\alpha$ and $s$ be an AFLO partition. Then the ascent partition $AP(r,s)$ is an AFLO partition.

\end{sublemma}

\begin{subdefi}[restriction]
If $r$ is a linear order on $\alpha$ and $E$ is in $AFLOPowerset$, we define the image of $E$ by the restriction map as an element
of $AFLOPowerset$: we take $powersetToPreorder(E)$, apply the function $AP(r,\cdot)$ and then apply $preorderToPowerset$.

\end{subdefi}

\begin{sublemma}[restriction\_is\_smaller]
If $r$ is a linear order on $\alpha$ and $E$ is in $AFLOPowerset$, the image of $E$ by the restriction map is contained in $E$.

\end{sublemma}

\begin{subdefi}[R]
If $r$ is a linear order on $\alpha$, we define a map $R$ from the set of facets of the Coxeter complex to the set of finite sets of
subsets of $\alpha$ by restricting the restriction map that we just defined.

\end{subdefi}


\subsection{The case of a finite set}

From now, we assume that $\alpha$ is finite.

\begin{sublemma}[AFLOPartitions\_is\_everything]
Let $s$ be a preorder on $\alpha$. Then $s$ is an AFLO partition if and only if it is a linearly ordered partition (i.e. a total preorder).

\end{sublemma}

\begin{sublemma}[AFLOPowerset\_is\_everything]
Let $E$ be a finite set of subsets of $\alpha$. Then $E$ is in AFLOPowerset if and only if it is totally ordered by inclusion and does not contain
$\varnothing$ and $\alpha$.

\end{sublemma}

\begin{sublemma}[Facets\_are\_linear\_orders]
Let $s$ be face of the Coxeter complex. Then $s$ is a facet if and only $powersetToPreorder(s)$ is a linear order on $\alpha$.

\end{sublemma}

\begin{sublemma}[R\_eq\_empty\_iff]
Let $r$ be a linear order on $\alpha$ and $s$ be a facet of the Coxeter complex. Then $R(s)=\varnothing$ if and only if
$preorderToPowerset(s)=r$.

\end{sublemma}


\begin{sublemma}[R\_eq\_self\_iff]
Let $r$ be a linear order on $\alpha$ and $s$ be a facet of the Coxeter complex. Then $R(s)=s$ if and only if
$preorderToPowerset(s)$ is the dual of $r$.

\end{sublemma}


\subsection{The distinguished facet map}


\section{Finite weighted complex}

We fix a set $\alpha$ and a function $\mu:\alpha\to\R$.

\subsection{Generalities about $\mu$}

\begin{sublemma}[Exists\_LinearOrder\_antitone]
There exists a linear order on $\alpha$ that makes the function $\mu$ antitone. 

\end{sublemma}


\begin{sublemma}[Positivity\_condition]
If $\alpha$ is finite, then $\mu$ is summable and the sum of $\mu$ is nonnegative if and only if it is
nonnegative as a finite sum.

\end{sublemma}

From now on, we assume that $\mu$ is summable and the sum of $\mu$ is nonnegative.

\subsection{Positive sets}

\begin{subdefi}[IsPositiveSet]
Let $X$ be a finite subset of $\alpha$. We say that $X$ is a \emph{positive set} if $\sum_{a\in X}\mu(a)\geq 0$.

\end{subdefi}

\begin{subdefi}[AFLOPowerset\_positive]
Definition of the set of positive elements of AFLOPowerset: these are the $E$ such that every element of $E$ is a positive set.

\end{subdefi}

\begin{subdefi}[AFLOPowerset\_positive.PartialOrder]
Definition of the partial order on the set of positive elements of AFLOPowerset, by lifting the partial order given by inclusion.

\end{subdefi}

\begin{sublemma}[AFLOPartitions\_forget\_positive]
The set of positive elements of AFLOPowerset is included in AFLOPowerset.

\end{sublemma}



\begin{subdefi}[AFLOPartitions\_positive]
The set of positive AFLO partitions: these are AFLO partitions whose image by powersetToPreorder is a positive element of AFLOPowerset.

\end{subdefi}

\begin{subdefi}[AFLOPartitions\_positive.PartialOrder]
The partial order on positive AFLO partitions obtained by lifting the partial order on preorders.

\end{subdefi}

\begin{sublemma}[AFLOPartitions\_forget\_positive]
The set of positive AFLO partitions is a subset of the set of AFLO partitions.

\end{sublemma}

\begin{sublemma}[AFLOPowerset\_positive\_down\_closed]
The set of positive elements of AFLOPowerset is down-closed (in the set of all finite sets of subsets of $\alpha$).

\end{sublemma}


\subsection{The weighted complex}

\begin{subdefi}[WeightedComplex]
The weighted complex is the abstract simplicial complex on the powerset of $\alpha$ whose faces are nonempty positive elements of AFLOPowerset.

\end{subdefi}

\begin{sublemma}[FacesWeightedComplex]
Let $s$ be a finite set of subsets of $\alpha$. Then $s$ is a face of the weighted complex if and only it is a nonempty positive element of AFLOPowerset.
    
\end{sublemma}

\begin{sublemma}[WeightedComplex\_subcomplex]
The weighted complex is a subcomplex of the Coxeter complex.    

\end{sublemma}

\begin{sublemma}[AFLO\_positive\_powersetToPreorder]
If $E$ is a positive element of AFLOPowerset, then $powersetToPreorder(E)$ is a positive AFLO partition.

\end{sublemma}

\begin{sublemma}[AFLO\_positive\_preorderToPowerset]
If $s$ is a positive AFLO partition, then $preorderToPowerset(s)$ is a positive element of AFLOPowerset.

\end{sublemma}

\begin{subdefi}[WeightedComplextoPositivePartitions]
The isomorphism of ordered sets between the set of positive elements of AFLOPowerset and the dual of the set of positive AFLO partitions,
given by the maps $powersetToPreorder$ and $preorderToPowerset$.

\end{subdefi}

\begin{sublemma}[WeightedFaces\_powersetToPreordertoPower]
If $s$ is a face of the weighted complex, then
$s=preorderToPowerset(powersetToPreorder(s))$.

\end{sublemma}

\begin{sublemma}[AFLOPartitions\_positive\_IsUpperSet]
Positive AFLO partial form an upper set (in the set of all preorders on $\alpha$).

\end{sublemma}


\subsection{The restriction map}

\begin{sublemma}[AscentPartition\_respects\_AFLO\_positive]
If $r$ is a linear order on $\alpha$ and $s$ is a positive AFLO partition, then the ascent partition $AP(r,s)$ is a positive AFLO partition.

\end{sublemma}

\begin{subdefi}[restriction\_weighted]
If $r$ is a linear order on $\alpha$, we define the restriction map from the set of positive elements of AFLOPowerset to itself: it sends
$E$ to $preorderToPowerset(AP(r,powersetToPreorder(E)))$.
    
\end{subdefi}

\begin{sublemma}[restriction\_weighted\_is\_smaller]
If $r$ is a linear order on $\alpha$ and $E$ is a positive element of AFLOPowerset, then the image of $E$ by the restriction map is included
in $E$.

\end{sublemma}

\begin{subdefi}[R\_weighted]
The restriction map as a map from the set of facets of the weighted complex to the set of finite sets of subsets of $\alpha$.

\end{subdefi}

\subsection{The case of a finite set}

From now, we suppose that $\alpha$ is finite.

\begin{sublemma}[AFLOPartitions\_iff]
Let $p$ be a preorder on $\alpha$. Then $p$ is a positive AFLO partition if and only if it is total and every element of $preorderToPowerset(p)$ is
a positive set.

\end{sublemma}

\begin{sublemma}[AFLOPowerset\_positive\_iff]
Let $E$ be a finite set of subsets of $\alpha$. Then $E$ is a positive element of AFLOPowerset if and only if it is totally ordered by inclusion,
doesn't contain $\varnothing$ and $\alpha$ and all its elements are positive sets.

\end{sublemma}

\begin{sublemma}[WeightedComplex\_nonempty\_iff]
The weighted complex is nonempty if and only if $card(\alpha)\geq 2$.

\end{sublemma}

\begin{sublemma}[WeightedComplex\_all\_iff]
The weighted complex is equal to the Coxeter complex if and only if $card(\alpha)<2$ or $\mu(a)\geq 0$ for every element $a$ of $\alpha$.
\footnote{This should also be true for infinite $\alpha$.}

\end{sublemma}


\subsection{The distinguished facet map}

\begin{sublemma}[LinearOrder\_etc\_respects\_AFLO\_positive]
Let $r$ be a linear order on $\alpha$ such that $\mu$ is antitone, and let $s$ be a positive AFLO partition. Then the associated linear order
$LO(r,s)$ is a positive AFLO partition.

\end{sublemma}

\begin{subdefi}[distinguishedFacet\_weighted]
If $r$ is a linear order on $\alpha$ such that $\mu$ is antitone, we define the distinguished facet map from the set of positive elements of 
AFLOPowerset to itself: it sends $E$ to $preorderToPowerset(LO(r,powersetToPreorder(E)))$.
    
\end{subdefi}

\begin{sublemma}[distinguishedFacet\_weighted\_is\_smaller]
If $r$ is a linear order on $\alpha$ such that $\mu$ is antitone and $E$ is a positive element of AFLOPowerset, then the image of $E$ by the 
distinguished facet map contains $E$.

\end{sublemma}

\begin{sublemma}[distinguishedFacet\_weighted\_is\_facet\_CoxeterComplex]
If $r$ is a linear order on $\alpha$ such that $\mu$ is antitone and $E$ is a facet of the weighted complex, then the image of $E$ by
the distinguished facet map is a facet of the Coxeter complex.

\end{sublemma}

\begin{sublemma}[distinguishedFacet\_weighted\_is\_face\_WeightedComplex]
If $r$ is a linear order on $\alpha$ such that $\mu$ is antitone and $E$ is a facet of the weighted complex, then the image of $E$ by
the distinguished facet map is a face of the weighted complex.

\end{sublemma}

\begin{sublemma}[FacetWeightedComplex\_iff]
Let $s$ be a finite set of subsets of $\alpha$. Then $s$ is a facet of the weighed complex if and only if it is a facet of the Coxeter complex
and a face of the weighted complex.

\end{sublemma}

\begin{sublemma}[R\_comparison]
If $r$ is a linear order on $\alpha$ and $s$ is a facet of the weighted complex, then its image by the maps $R$ for the Coxeter complex and
the weighted complex are equal.

\end{sublemma}


\begin{subdefi}[DF\_weighted]
If $r$ is a linear order on $\alpha$ such that $\mu$ is antitone,
the distinguished facet map as a map from the set of faces of the weighted complex to the set of facets of the weighted complex.

\end{subdefi}


\begin{sublemma}[DF\_comparison]
If $r$ is a linear order on $\alpha$ such that $\mu$ is antitone and $s$ is a face of the weighted complex, then its image by the distinguished
facet maps for the Coxeter complex and the weighted complex are equal.

\end{sublemma}

\begin{sublemma}[R\_weighted\_eq\_empty\_iff]
Let $r$ be a linear order on $\alpha$ and $s$ be a facet of the weighed complex. Then $R(s)=\varnothing$ if and only if $powersetToPreorder(s)=r$.

\end{sublemma}

\begin{sublemma}[R\_weighted\_eq\_self\_iff]
Let $r$ be a linear order on $\alpha$ such that $\mu$ is antitone and $s$ be a facet of the weighed complex. Then $R(s)=s$ if and only if $powersetToPreorder(s)$ is 
the dual of $r$.

\end{sublemma}


\subsection{Facets of the weighted complex}

\begin{sublemma}[Fixed\_linear\_order\_in\_AFLO\_positive]
If $r$ is a linear order on $\alpha$ such that $\mu$ is antitone, then $r$ is a positive AFLO partition.

\end{sublemma}

\begin{sublemma}[Fixed\_linear\_order\_in\_WeightedComplex]
If $r$ is a linear order on $\alpha$ such that $\mu$ is antitone and if $card(\alpha)\geq 2$, then $preorderToPowerset(r)$ is a face of the weighted complex.

\end{sublemma}

\begin{sublemma}[Dual\_linear\_order\_in\_AFLO\_positive]
If $r$ is a linear order on $\alpha$ such that $\mu$ is antitone, then the dual $r$ is a positive AFLO partition if and only $\mu(a)\geq 0$ for every element
$a$ of $\alpha$.

\end{sublemma}

\begin{sublemma}[Dual\_linear\_order\_in\_WeightedComplex]
If $r$ is a linear order on $\alpha$ such that $\mu$ is antitone and if $card(\alpha)\geq 2$, then the image of the dual $r$ by $preorderToPowerset$ is a 
face of the weighted complex if and only $\mu(a)\geq 0$ for every element

\end{sublemma}


\subsection{Decomposability of the weighted complex}

\begin{sublemma}[WeightedComplex\_is\_decomposable]
If $r$ is a linear order on $\alpha$ such that $\mu$ is antitone, then the pair $(R,DF)$ is a decomposition of the weighted complex.

\end{sublemma}

\subsection{Some properties of the weighted complex}

\begin{sublemma}[WeightedComplex\_is\_finite]
The weighted complex is finite.

\end{sublemma}

\begin{sublemma}[WeightedComplex\_dimension\_facet]
If $s$ is a facet of the weighted complex, then $card(s)=card(\alpha)-1$.

\end{sublemma}

\begin{sublemma}[WeightedComplex\_is\_pure]
The weighted complex is pure.

\end{sublemma}

\begin{sublemma}[WeightedComplex\_Pi0Facet]
Let $r$ be a linear order on $\alpha$ such that $\mu$ is antitone and let $s$ be a facet of the weighted complex. Then $s$ is a $\pi_0$ facet if
and only if $powersetToPreorder(s)=r$ or $card(\alpha)=2$.

\end{sublemma}

\begin{sublemma}[WeightedComplex\_HomologyFacet]
Let $r$ be a linear order on $\alpha$ such that $\mu$ is antitone and let $s$ be a facet of the weighted complex. Then $s$ is a homology facet if
and only if $powersetToPreorder(s)$ is equal to the dual of $r$ and $card(\alpha)>2$.

\end{sublemma}


\subsection{Shellability of the weighted complex}

\begin{subdefi}[WeightedComplexFacets\_to\_LinearOrders]
A map from the set of facets of the weighted complex to the set of linear orders on $\alpha$ (given by $powersetToPreorder$).

\end{subdefi}

\begin{sublemma}[WeightedComplexFacets\_to\_LinearOrders\_injective]
The map of the previous definition is injective.

\end{sublemma}

\begin{subdefi}[WeakBruhatOrder\_facets\_WeightedComplex]
Let $r$ be a linear order on $\alpha$. We define the weak Bruhat order (relative to $r$) on the set of facets of the weighted complex by
lifting the weak Bruhat on the set of linear orders on $\alpha$.

\end{subdefi}

\begin{sublemma}[WeakBruhat\_compatible\_with\_DF\_weighted]
Let $r$ be a linear order on $\alpha$ such that $\mu$ is antitone. Then the weak Bruhat order on the set of facets of the weighted complex
is compatible with the map $DF$.

\end{sublemma}

\begin{sublemma}[WeightedComplexShelling]
Let $r$ be a linear order on $\alpha$ such that $\mu$ is antitone. Then every linear order on the set of facets of the weighted complex refining
the weak Bruhat order is a shelling order.

\end{sublemma}


\subsection{Euler-Poincaré characteristic of the weighted complex}

\begin{subdefi}[FacetWeightedComplexofLinearOrder]
Let $r$ be a linear order on $\alpha$ such that $\mu$ is antitone. If $card(\alpha)\geq 2$, this defines the facet of the weighted complex
corresponding to $preorderToPowerset(r)$.

\end{subdefi}

\begin{sublemma}[WeightedComplex\_of\_pair]
Let $a,b$ be elements of $\alpha$ such that $a\ne b$ and $\alpha$ is equal to $\{a,b\}$.  If $\mu(a)\geq 0$ and $\mu(b)<0$, then a finite set
of subsets of $\alpha$ is a face of the weighted complex if and only it is equal to $\{\{a\}\}$.

\end{sublemma}

\begin{sublemma}[preorderToPowerset\_of\_pair]
Let $r$ be a linear order on $\alpha$ and $a,b$ be elements of $\alpha$ such that $a<_r b$ and $\alpha$ is equal to $\{a,b\}$.  Then 
$preorderToPowerset(r)$ is equal to $\{\{a\}\}$.

\end{sublemma}

\begin{sublemma}[EulerPoincareCharacteristic\_WeightedComplex]
Suppose that $\alpha$ is nonempty. Then the Euler-Poincaré characteristic of the weighted complex is equal to $1+(-1)^{card(\alpha)}$ if
$\mu(a)\geq 0$ for every element $a$ of $\alpha$, and to $1$ otherwise. 

\end{sublemma}



\section{Application}

We fix a finite set $\alpha$.

\subsection{Some lemmas}

\begin{subdefi}[fintypeLinearOrderedPartitions]
The set of linearly ordered partitions of $\alpha$ is finite.

\end{subdefi}

\begin{subdefi}[fintypeAntisymmetrization]
For every preorder $s$ on $\alpha$, the antisymmetrization of $s$ is finite.

\end{subdefi}

\begin{subdefi}[CardBlocksPartition]
For $s$ a preorder on $\alpha$, the cardinality of the set of blocks of $s$ is the cardinality of the antisymmetrization of $s$.

\end{subdefi}

\begin{sublemma}[CardBlocksTrivialPreorder\_nonempty]
If $\alpha$ is nonempty, then the cardinality of the set of blocks of the trivial preorder is equal to $1$.

\end{sublemma}

\begin{sublemma}[CardBlocksTrivialPreorder\_empty]
If $\alpha$ is empty, then the cardinality of the set of blocks of the trivial preorder is equal to $0$.

\end{sublemma}

\begin{sublemma}[CardBlocksPartition\_vs\_card\_preorderToPowerset]
If $\alpha$ is nonempty, then, for every preorder $s$ on $\alpha$, the cardinality of the set of blocks of $s$ is equal to 
$card(preorderToPowerset(s))+1$.

\end{sublemma}

\begin{sublemma}[CardBlocksTwoStepPreorder]
If $a,b$ are elements of $\alpha$ such that $a\ne b$, then cardinality of the set of blocks of the two-step preorder defined by $a$
is equal to $2$.

\end{sublemma}


\subsection{Alternating sum on positive linearly ordered partitions}

We fix a function $\mu:\alpha\to\R$ such that $\sum_{\alpha}\mu\geq 0$.

\begin{sublemma}[AFLOPartitions\_positive\_eq]
The set of positive AFLO partitions is equal to the set of total preorders $s$ such that, for every element $a$ of $\alpha$,
the half-infinite interval $]\leftarrow,a]$ for $s$ is a positive set. 

\end{sublemma}

\begin{sublemma}[trivialPreorder\_in\_AFLOPartitions\_positive]
The trivial preorder is a positive AFLO partition.

\end{sublemma}

\begin{sublemma}[AFLOPartitions\_positive\_of\_empty]
If $\alpha$ is empty, then the set of positive AFLO partitions is the singleton consisting of the trivial preorder.

\end{sublemma}

\begin{sublemma}[AFLOPartitions\_positive\_of\_singleton]
If there exists an element $a$ of $\alpha$ such that all elements of $\alpha$ are equal to $a$.
then the set of positive AFLO partitions is the singleton consisting of the trivial preorder.

\end{sublemma}

\begin{subdefi}[Sum\_of\_signs\_over\_AFLOPartitions\_positive]
We define the sum that we want to calculate: this is the sum over positive AFLO partitions of $-1$ to
the cardinality of the set of blocks of $s$.

\end{subdefi}

\begin{sublemma}[Sum\_of\_signs\_eq]
We calculate the sum of the previous definition: it is equal to $(-1)^{card(\alpha)}$ if $\mu(a)\geq 0$ for every element $a$ of $\alpha$,
and to $0$ otherwise.

\end{sublemma}

\end{document}