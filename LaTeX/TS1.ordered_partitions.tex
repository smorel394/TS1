\section{Linearly ordered partitions}

We fix a set $\alpha$.

\begin{subdefi}[dual]
If $r$ is a linear order on $\alpha$, then $dual(r)$ is the dual linear order.

\end{subdefi}


\subsection{The partial order on linearly ordered partitions}

\begin{subdefi}[LinearOrderedPartitions]
A linearly ordered partitions on $\alpha$ is a total preorder on $\alpha$.

\end{subdefi}

\begin{subdefi}[LinearOrderedPartitions.PartialOrder]
We restrict the partial order on preorders to a partial order on linearly ordered partitions.

\end{subdefi}

\begin{sublemma}[trivialPreorder\_is\_greatest\_partition]
The trivial preorder is the greatest element of the set of linearly ordered partitions.

\end{sublemma}

\begin{sublemma}[linearOrder\_is\_minimal\_partition]
A linear order is minimal in the set of linearly ordered partitions.

\end{sublemma}


\subsection{Cooking up a linear order from a total preorder}

\begin{subdefi}[LinearOrder\_of\_total\_preorder\_and\_linear\_order\_aux]
Let $r$ be a linear order on $\alpha$ and $s$ be a preorder on $\alpha$. We define a relation
$LO(r,s)$ on $\alpha$ by saying that $a\le_{LO(r,s)}b$ if and only if $a<_s b$, or $a\simeq_s b$ and
$a\le_r b$. 

\end{subdefi}

The idea is that we want to define a linear order smaller than or equal to $s$, and we use $r$ to order the elements
of $\alpha$ that are equivalent for $s$.

\begin{sublemma}[LinearOrder\_of\_total\_preorder\_and\_linear\_order\_aux \& LinearOrder\_of\_total\_preorder\_and\_linear\_order\_refl \& LinearOrder\_of\_total\_preorder\_and\_linear\_order\_trans
\& LinearOrder\_of\_total\_preorder\_and\_linear\_order\_antisymm \& LinearOrder\_of\_total\_preorder\_and\_linear\_order\_total]
Let $r$ be a linear order on $\alpha$ and $s$ be a total preorder on $\alpha$. Then $LO(r,s)$ is a linear order on $\alpha$.

\end{sublemma}

\begin{subdefi}[LinearOrder\_of\_total\_preorder\_and\_linear\_order]
The relation $LO(r,s)$ as a preorder.

\end{subdefi}

\begin{sublemma}[LinearOrder\_of\_total\_preorder\_and\_linear\_order\_is\_total]
The preorder $LO(r,s)$ is total if $s$ is total.

\end{sublemma}

\begin{sublemma}[LinearOrder\_of\_total\_preorder\_and\_linear\_order\_is\_linear]
The preorder $LO(r,s)$ is a linear order if $s$ is total.

\end{sublemma}

\begin{sublemma}[LinearOrder\_of\_total\_preorder\_and\_linear\_order\_is\_smaller]
The preorder $LO(r,s)$ is smaller than or equal to $s$.

\end{sublemma}

\begin{sublemma}[LinearOrder\_vs\_fixed\_LinearOrder]
If $s$ is total and $a,b$ are elements of $\alpha$ such that $a\simeq_s b$, then
$a\le_r b$ if and only if $a\le_{LO(r,s)}b$.

\end{sublemma}

\begin{sublemma}[LinearOrder\_of\_total\_preorder\_and\_linear\_order\_lt]
Suppose that $s$ is total, and let $a,b$ be elements of $\alpha$. If $a<_{LO(r,s)}b$, then $a<_s b$.

\end{sublemma}

\begin{sublemma}[LinearOrder\_of\_linear\_order\_and\_linear\_order\_is\_self]
If $s$ is a linear order, then $LO(r,s)=s$.

\end{sublemma}

\begin{sublemma}[minimal\_partition\_is\_linear\_order]
A minimal linearly ordered partition is a linear order.

\end{sublemma}

We finish this subsection with some lemmas about the principal lower sets of $LO(r,s)$.

\begin{sublemma}[LowerSet\_LinearOrder\_etc\_is\_disjoint\_union]
Let $a$ be an element of $\alpha$ and $X$ be the half-infinite interval $]\leftarrow,a]$ for the preorder $LO(r,s)$. Then
$X$ is the disjoint union of the half-infinite interval $]\leftarrow,a[$ for the preorder $s$ and of the set
$\{b| b\le_r a\mbox{ and }a\simeq_s b\}$.

\end{sublemma}

\begin{sublemma}[LowerSet\_LinearOrder\_etc\_is\_difference]
Let $a$ be an element of $\alpha$ and $X$ be the half-infinite interval $]\leftarrow,a]$ for the preorder $LO(r,s)$. Then
$X$ is the difference of the half-infinite interval $]\leftarrow,a]$ for the preorder $s$ and of the set
$\{b| a<_r b\mbox{ and }a\simeq_s b\}$, and the second of these sets is contained in the first.

\end{sublemma}


\subsection{The ascent partition of a preorder}

\begin{subdefi}[AscentPartition\_aux]
If $r$ is a linear order on $\alpha$ and $s$ is a preorder on $\alpha$, the \emph{ascent partition} of $s$ (with respect to $r$)
is the relation $AP(r,s)$ defined by $a\le_{AP(r,s)}b$ if $a\le_s b$, or $b\le_s a$ and the identity on the interval $[b,a]$ for $s$
is striclty monotone for the preorders $s$ and $r$.
(The last condition means that, if $c,d$ are elements of $\alpha$ such that $a\le_s c <_s d \le_s b$, then $c<_r d$.)

\end{subdefi}

\begin{sublemma}[AscentPartition\_aux\_refl \& AscentPartition\_aux\_trans \& AscentPartition\_aux\_total]
If $s$ is a total preorder, then $AP(r,s)$ is a total preorder.

\end{sublemma}

\begin{subdefi}[AscentPartition]
The relation $AP(r,s)$ as a preorder (for $s$ a total preorder).

\end{subdefi}

\begin{sublemma}[AscentPartition\_is\_total]
If $s$ is a total preorder, then the preorder $AP(r,s)$ is total.

\end{sublemma}

\begin{sublemma}[AscentPartition\_is\_greater]
If $s$ is a total preorder, then $AP(r,s)$ is greater than or equal to $s$.

\end{sublemma}


\subsection{Interactions between these two constructions}

\begin{sublemma}[AscentPartition\_comp]
If $r$ is a linear order and $s$ is a total preorder, then $AP(r,s)=AP(r,LO(r,s))$.

\end{sublemma}

\begin{sublemma}[LinearOrder\_of\_AscentPartition]
If $r$ is a linear order and $s$ is a total preorder, then $LO(r,s)=LO(r,AP(r,s))$.

\end{sublemma}

\begin{sublemma}[LinearOrder\_of\_total\_preorder\_and\_linear\_order\_is\_constant\_on\_interval\_aux \& LinearOrder\_of\_total\_preorder\_and\_linear\_order\_is\_constant\_on\_interval]
Let $r$ be a linear order on $\alpha$ and $s,t,u$ be total preorders on $\alpha$ such that $s\le t\le u$ and $LO(r,s)=LO(r,u)$. Then $LO(r,s)=LO(r,t)$.

\end{sublemma}

\begin{sublemma}[LinearOrder\_of\_total\_preorder\_and\_linear\_order\_on\_ascent\_interval]
Let $r$ be a linear order on $\alpha$ and $s,t$ be total preorders on $\alpha$ such that $s\le t\le AP(r,s)$. Then $LO(r,s)=LO(r,t)$.  

\end{sublemma}


\begin{sublemma}[LinearOrder\_of\_total\_preorder\_and\_linear\_order\_on\_ascent\_interval']
Let $r$ be a linear order on $\alpha$ and $s,t$ be total preorders on $\alpha$ such that $s\le t\le AP(r,s)$ and $s$ is a linear order. Then $s=LO(r,t)$.  

\end{sublemma}

\begin{sublemma}[LinearOrder\_of\_total\_preorder\_and\_linear\_order\_fibers]
Let $r$ be a linear order on $\alpha$ and $s,t$ be preorder on $\alpha$ such that $s$ is a linear order, $t$ is total and $LO(r,t)=s$. Then we have
$s\le t\le AP(r,s)$.

\end{sublemma}

\begin{sublemma}[AscentPartition\_fibers]
Let $r$ be a linear order on $\alpha$ and $s,t$ be preorder on $\alpha$ such that $s$ is a linear order and $t$ is total.
Then $AP(r,s)=AP(r,t)$ if and only if $s\le t\le AP(r,s)$.

\end{sublemma}

\begin{sublemma}[AscentPartition\_fibers']
Let $r$ be a linear order on $\alpha$ and $s,t$ be preorder on $\alpha$ such that $s$ is a linear order and $t$ is total.
Then $AP(r,s)=AP(r,t)$ if and only if $s=LO(r,t)$.

\end{sublemma}


\subsection{Eventually trivial partitions}

This part does not seem to be useful anymore.

\begin{subdefi}[EventuallyTrivialLinearOrderedPartitions]
Let $s$ be a linear order on $\alpha$. A linearly ordered partitions $s$ is called \emph{essentially trivial} if there exists
$a$ in $\alpha$ such that, for all $b,c$ in $\alpha$, if $b,c\ge_r a$, we have $b\le_s c$.

\end{subdefi}

\begin{sublemma}[EventuallyTrivial\_is\_finite]
Let $r$ be a linear order on $\alpha$ that is locally finite with a smallest element, and let $s$ be an essentially trivial linearly
ordered partition. Then the antisymmetrization of $s$ is finite.

\end{sublemma}

\begin{sublemma}[EventuallyTrivial\_IsUpperSet]
Let $r$ be a linear order on $\alpha$. Then eventually trivial linearly ordered partitions form an upper set.

\end{sublemma}

\begin{sublemma}[Finite\_is\_EventuallyTrivial]
If $\alpha$ is finite nonempty, then every linearly ordered partition is eventually trivial (for any choice of linear order $r$).

\end{sublemma}


\subsection{Some calculations}

\begin{sublemma}[AscentPartition\_fixed\_linear\_order]
Let $r$ be a linear order on $\alpha$. Then $AP(r,r)$ is the trivial preorder on $\alpha$.

\end{sublemma}

\begin{sublemma}[Preorder\_lt\_and\_AscentPartition\_ge\_implies\_LinearOrder\_le]
Let $r$ be a linear order and $s$ be a total preorder. For all $a,b$ in $\alpha$ such that
$a<_s b$ and $b\le_{AP(r,s)} a$, we have $a\le_r b$.

\end{sublemma}

\begin{sublemma}[AscentPartition\_trivial\_implies\_fixed\_linear\_order]
Let $r$ be a linear order and $s$ be a preorder. If $s$ is a linear order and $AP(r,s)$ is the trivial preorder, then $s=r$. 

\end{sublemma}

\begin{sublemma}[AscentPartition\_dual\_fixed\_linear\_order]
Let $r$ be a linear order on $\alpha$. Then $AP(r, dual(r))$ is equal to $dual(r)$.

\end{sublemma}

We want to prove the converse of the last lemma under some conditions on $s$. This requires some preparation.

\begin{subdefi}[ReverseProductOrder]
If $s$ is a preorder on $\alpha$, then this is the preorder on $\alpha\times\alpha$ that is the product of the dual of $\alpha$ and of $\alpha$.

\end{subdefi}

\begin{sublemma}[ReverseProductOrder\_lt1 \& ReverseProductOrder\_lt2]
Let $s$ be a preorder on $\alpha$ and $a,b,c$ be elements of $\alpha$. If $a<_s b$, then $(b,c)<(a,c)$ for the reverse product order, and if
$b<_s c$, then $(a,b)<(a,c)$ for the reverse product order.

\end{sublemma}

\begin{sublemma}[Exists\_smaller\_noninversion]
Let $r$ be a linear order and $s$ be a preorder. We suppose that $s$ is a linear order and that $AP(r,s)=s$. If $a,b$ are elements of $\alpha$
such that $a<_r b$ and $a<_s b$, then there exist $c,d$ in $\alpha$ such that $c<_r d$, $c<_s d$ and $(c,d)$ is strictly smaller than $(a,b)$ for
the reverse product preorder defined by $s$.

\end{sublemma}

\begin{sublemma}[AscentPartition\_linear\_implies\_dual\_linear\_order]
Let $r$ be a linear order and $s$ be a preorder. We suppose that $s$ is a linear order, that $AP(r,s)=s$ and that $s$ is locally finite.
Then is equal to the dual of $r$.

\end{sublemma}